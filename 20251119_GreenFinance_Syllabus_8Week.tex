\documentclass[8pt]{article}
\usepackage[margin=1in]{geometry}
\usepackage{enumitem}
\usepackage{hyperref}
\usepackage{booktabs}
\usepackage{longtable}

\title{\textbf{Green Finance Professional Certificate} \\ Intensive 8-Week Program}
\author{}
\date{}

\begin{document}

\maketitle

\section*{Course Overview}

\subsection*{Program Description}
This intensive professional certificate provides finance professionals with comprehensive knowledge and practical skills in green finance, sustainable investment, climate risk assessment, and impact measurement. The program covers the full spectrum from green bonds and ESG integration to renewable energy finance, regulatory frameworks, and emerging areas like natural capital finance. Designed to meet the surging market demand for green finance expertise, this program prepares participants for immediate career advancement in this rapidly growing field.

\subsection*{Program Details}
\begin{itemize}[leftmargin=*]
    \item \textbf{Duration:} 8 weeks intensive (108 contact hours)
    \item \textbf{Format:} In-person instruction with hands-on workshops
    \item \textbf{Schedule:} 13-14 hours per week
    \item \textbf{Target Audience:} Finance professionals transitioning to sustainable finance
    \item \textbf{Prerequisites:} None (beginner-friendly with comprehensive foundation building)
    \item \textbf{Credential:} University-issued Professional Certificate in Green Finance
    \item \textbf{Class Size:} Maximum 25 participants for optimal interaction
\end{itemize}

\subsection*{Market Context and Demand}
The green finance sector is experiencing explosive growth with a significant skills gap:
\begin{itemize}[leftmargin=*]
    \item Green hiring growing at 8\% per year, while green skills supply grows at only 4.3\%
    \item Financial services sector: 16.3\% spike in green hires (2023)
    \item 17,000+ green job vacancies with only 900 graduates possessing required sustainability skills
    \item Average salary premium: 15-25\% for green finance specialists
    \item This certificate directly addresses the identified skills gap for mid-career professionals
\end{itemize}

\subsection*{Learning Objectives}
Upon completion, participants will be able to:
\begin{enumerate}[leftmargin=*]
    \item Understand and apply core concepts, instruments, and markets in green finance
    \item Analyze and utilize ESG data for investment screening and portfolio construction
    \item Assess climate-related financial risks using TCFD and scenario analysis frameworks
    \item Develop financial models for renewable energy projects and green investments
    \item Navigate global regulatory frameworks including EU Taxonomy, SFDR, and emerging standards
    \item Evaluate and structure green bonds and sustainable debt instruments
    \item Apply impact investing principles and measure social and environmental returns
    \item Understand biodiversity finance and natural capital frameworks (TNFD)
    \item Use practical tools (Python, Excel, ESG platforms) for green finance analysis
    \item Integrate green finance concepts across organizational functions
\end{enumerate}

\subsection*{Assessment Components}
\begin{itemize}[leftmargin=*]
    \item Weekly Assignments and Case Studies (4 assignments): 30\%
    \item Midterm Financial Modeling Project: 20\%
    \item Research Paper: 20\%
    \item Final Integration Project and Presentation: 30\%
\end{itemize}

\subsection*{Required Tools and Software}
\begin{itemize}[leftmargin=*]
    \item Microsoft Excel (financial modeling)
    \item Python 3.x with pandas, numpy, matplotlib libraries (provided setup guide)
    \item Access to ESG data platforms (institutional access provided during course)
    \item PDF reader and presentation software
\end{itemize}

\newpage

\section*{Weekly Schedule}

\subsection*{Week 1: Green Finance Foundations (14 hours)}

\textbf{Learning Objectives:}
\begin{itemize}[leftmargin=*]
    \item Define green finance and understand its evolution and market structure
    \item Identify key stakeholders, instruments, and drivers in the green finance ecosystem
    \item Understand the climate imperative and financial risks/opportunities
    \item Apply fundamental financial concepts (time value of money, bond pricing, portfolio theory) to green finance
\end{itemize}

\textbf{Session Breakdown:}

\textit{Session 1.1 - Introduction to Green Finance (3 hours)}
\begin{itemize}[leftmargin=*]
    \item Definition, scope, and evolution of green finance
    \item Climate change and the financial system linkage
    \item Market size, growth trajectory, and investment gap (\$2-3 trillion annually)
    \item Paris Agreement Article 2.1c implications for finance
\end{itemize}

\textit{Session 1.2 - Green Finance Ecosystem (3 hours)}
\begin{itemize}[leftmargin=*]
    \item Market participants: capital providers, recipients, intermediaries
    \item Role of development finance institutions and blended finance
    \item International organizations and standard-setters (ICMA, FSB, NGFS, ISSB)
    \item Public-private partnerships in green finance
\end{itemize}

\textit{Session 1.3 - Green Financial Instruments Overview (4 hours)}
\begin{itemize}[leftmargin=*]
    \item Green bonds, sustainability-linked instruments, and green loans
    \item Carbon markets: compliance (EU ETS) and voluntary markets
    \item Green equity, funds, and ETFs
    \item Emerging instruments: blue bonds, transition bonds
    \item Performance evidence: returns, risks, and greenium
\end{itemize}

\textit{Session 1.4 - Financial Fundamentals Workshop (4 hours)}
\begin{itemize}[leftmargin=*]
    \item Time value of money and NPV in green projects
    \item Bond pricing, yields, and duration
    \item Portfolio theory and diversification
    \item Risk-return analysis for green investments
    \item Excel modeling foundations
\end{itemize}

\textbf{Readings:}
\begin{itemize}[leftmargin=*]
    \item Berrou et al. (2019). ``A taxonomy of green finance''
    \item Giglio, Kelly, Stroebel (2021). ``Climate Finance'' (Annual Review)
    \item Climate Bonds Initiative (2024). ``State of the Market Report''
    \item UNEP (2024). ``Global Landscape of Climate Finance''
\end{itemize}

\textbf{Assignment:} None (foundation week)

\subsection*{Week 2: Green Bonds and Sustainable Debt Markets (14 hours)}

\textbf{Learning Objectives:}
\begin{itemize}[leftmargin=*]
    \item Understand green bond market structure, principles, and issuance process
    \item Analyze green bond frameworks and assess alignment with standards
    \item Evaluate pricing dynamics and the greenium phenomenon
    \item Differentiate between green bonds, sustainability-linked instruments, and transition bonds
    \item Structure and price green debt instruments
\end{itemize}

\textbf{Session Breakdown:}

\textit{Session 2.1 - Green Bond Markets and Principles (3.5 hours)}
\begin{itemize}[leftmargin=*]
    \item Market history: EIB 2007 to \$1.6 trillion market today
    \item Green Bond Principles (ICMA): four core components
    \item Use of proceeds categories and eligible projects
    \item Market segmentation: sovereign, corporate, municipal, supranational
    \item Geographic trends: Europe, Asia, Americas
\end{itemize}

\textit{Session 2.2 - Green Bond Issuance and Verification (3.5 hours)}
\begin{itemize}[leftmargin=*]
    \item Framework development: structure and requirements
    \item External reviews: second-party opinions, verification, certification
    \item Role of verifiers (Sustainalytics, CICERO, Vigeo Eiris)
    \item Reporting requirements and impact measurement
    \item Case study: Analyzing a major corporate green bond issuance
\end{itemize}

\textit{Session 2.3 - Pricing, Performance, and the Greenium (3.5 hours)}
\begin{itemize}[leftmargin=*]
    \item Green bond pricing dynamics
    \item Greenium: evidence, magnitude (-2 to -5 bps), and drivers
    \item Performance comparison: green vs. conventional bonds
    \item Liquidity considerations and market efficiency
    \item Investor demand and oversubscription patterns
\end{itemize}

\textit{Session 2.4 - Beyond Green Bonds: SLBs, Transition, and Innovation (3.5 hours)}
\begin{itemize}[leftmargin=*]
    \item Sustainability-Linked Bonds: KPIs, SPTs, and step-up mechanisms
    \item Transition bonds and Climate Transition Finance Handbook (ICMA)
    \item Social bonds and sustainability bonds
    \item Greenwashing risks and controversies
    \item Workshop: Green bond framework analysis and pricing exercise
\end{itemize}

\textbf{Readings:}
\begin{itemize}[leftmargin=*]
    \item ICMA (2024). ``Green Bond Principles''
    \item Flammer (2021). ``Corporate green bonds'' (JFE)
    \item Zerbib (2019). ``The effect of pro-environmental preferences on bond prices''
    \item Tang \& Zhang (2020). ``Do shareholders benefit from green bonds?''
    \item ICMA (2023). ``Climate Transition Finance Handbook''
\end{itemize}

\textbf{Assignment 1 Due (End of Week 2):} Green Bond Framework Evaluation and Pricing Analysis

\subsection*{Week 3: ESG Integration and Data Analytics (13 hours)}

\textbf{Learning Objectives:}
\begin{itemize}[leftmargin=*]
    \item Understand ESG rating methodologies and data providers
    \item Apply ESG screening, integration, and thematic investing approaches
    \item Use Python and ESG platforms for data analysis and portfolio construction
    \item Evaluate materiality and ESG performance metrics
    \item Critically assess ESG rating divergence and quality issues
\end{itemize}

\textbf{Session Breakdown:}

\textit{Session 3.1 - ESG Frameworks, Ratings, and Controversies (3 hours)}
\begin{itemize}[leftmargin=*]
    \item ESG concepts: environmental, social, governance dimensions
    \item Major rating providers: MSCI, Sustainalytics, Refinitiv, ISS ESG
    \item Rating methodologies and scoring approaches
    \item The divergence problem: why ratings disagree
    \item Data quality, coverage, and comparability challenges
\end{itemize}

\textit{Session 3.2 - ESG Integration Strategies (3 hours)}
\begin{itemize}[leftmargin=*]
    \item Negative screening and exclusions
    \item Best-in-class and positive screening
    \item ESG integration into fundamental analysis
    \item Thematic and impact investing approaches
    \item Materiality assessment: financial vs. impact materiality
    \item Engagement and active ownership
\end{itemize}

\textit{Session 3.3 - ESG Data Platforms Workshop (3 hours)}
\begin{itemize}[leftmargin=*]
    \item Hands-on: MSCI/Refinitiv platform navigation
    \item Data extraction, filtering, and export
    \item Company ESG profiles and controversy analysis
    \item Sector benchmarking and peer comparison
    \item Creating custom ESG screens
\end{itemize}

\textit{Session 3.4 - Python for ESG Portfolio Analytics (4 hours)}
\begin{itemize}[leftmargin=*]
    \item Python environment setup and pandas basics
    \item Loading and cleaning ESG datasets
    \item Portfolio screening based on ESG criteria
    \item ESG score calculations and weighted averages
    \item Visualization: ESG distribution, sector analysis
    \item Backtesting ESG-screened portfolios
    \item Performance attribution: ESG vs. financial factors
\end{itemize}

\textbf{Readings:}
\begin{itemize}[leftmargin=*]
    \item Berg et al. (2022). ``Aggregate confusion: The divergence of ESG ratings''
    \item Khan et al. (2016). ``Corporate sustainability: First evidence on materiality''
    \item Eccles \& Stroehle (2018). ``Exploring social origins in ESG''
    \item Dimson et al. (2015). ``Active ownership''
\end{itemize}

\textbf{Assignment 2 Due (End of Week 3):} Python ESG Portfolio Construction and Analysis

\subsection*{Week 4: Climate Risk Assessment and TCFD (13 hours)}

\textbf{Learning Objectives:}
\begin{itemize}[leftmargin=*]
    \item Understand physical and transition climate risks and their financial implications
    \item Apply TCFD framework for climate-related disclosure and risk assessment
    \item Conduct scenario analysis using NGFS and other climate scenarios
    \item Quantify climate risks: Climate VaR, carbon footprinting, stranded assets
    \item Integrate climate risk into investment and lending decisions
\end{itemize}

\textbf{Session Breakdown:}

\textit{Session 4.1 - Climate Risk Fundamentals (3.5 hours)}
\begin{itemize}[leftmargin=*]
    \item Physical risks: acute (extreme weather) and chronic (sea level rise, temperature)
    \item Transition risks: policy, technology, market, reputation
    \item Climate risk transmission to financial system
    \item Systemic risk and ``green swan'' events
    \item Sector-specific vulnerabilities
\end{itemize}

\textit{Session 4.2 - TCFD Framework and Disclosure (3.5 hours)}
\begin{itemize}[leftmargin=*]
    \item TCFD structure: governance, strategy, risk management, metrics \& targets
    \item Climate scenario analysis: methodology and best practices
    \item NGFS scenarios: orderly, disorderly, hot house world
    \item Case study: Analyzing corporate TCFD reports
    \item Regulatory momentum: mandatory TCFD adoption globally
\end{itemize}

\textit{Session 4.3 - Climate Risk Quantification Methods (3 hours)}
\begin{itemize}[leftmargin=*]
    \item Climate Value-at-Risk (CVaR) methodologies
    \item Carbon footprinting: Scope 1, 2, 3 emissions
    \item Stranded asset risk assessment
    \item Physical risk modeling tools and data providers
    \item Transition pathway analysis
    \item Implied Temperature Rise (ITR) metrics
\end{itemize}

\textit{Session 4.4 - Scenario Analysis Workshop (3 hours)}
\begin{itemize}[leftmargin=*]
    \item Hands-on: Portfolio climate risk assessment
    \item Excel-based scenario modeling
    \item Calculating portfolio carbon footprint
    \item Climate stress testing framework
    \item Group exercise: Sector-specific transition risk analysis
\end{itemize}

\textbf{Readings:}
\begin{itemize}[leftmargin=*]
    \item TCFD (2017). ``Final Report: Recommendations''
    \item Bolton et al. (2020). ``The green swan'' (BIS)
    \item Battiston et al. (2017). ``A climate stress-test of the financial system''
    \item Krueger et al. (2020). ``The importance of climate risks for institutional investors''
    \item NGFS (2024). ``Climate Scenarios for Central Banks and Supervisors''
\end{itemize}

\textbf{Midterm Project Checkpoint:} Climate risk analysis component due

\subsection*{Week 5: Renewable Energy Project Finance (13 hours)}

\textbf{Learning Objectives:}
\begin{itemize}[leftmargin=*]
    \item Understand renewable energy technologies, economics, and market trends
    \item Apply project finance principles to renewable energy investments
    \item Develop comprehensive financial models for solar and wind projects
    \item Evaluate risks and structure financing for clean energy projects
    \item Analyze power purchase agreements and revenue mechanisms
\end{itemize}

\textbf{Session Breakdown:}

\textit{Session 5.1 - Renewable Energy Technologies and Economics (3 hours)}
\begin{itemize}[leftmargin=*]
    \item Solar PV: technology evolution and cost decline (90\% reduction since 2010)
    \item Wind energy: onshore and offshore developments
    \item Energy storage: batteries and grid integration
    \item Emerging technologies: green hydrogen, geothermal, tidal
    \item Levelized Cost of Energy (LCOE) analysis
    \item Learning curves and technology forecasting
\end{itemize}

\textit{Session 5.2 - Project Finance Structure and Principles (3.5 hours)}
\begin{itemize}[leftmargin=*]
    \item Project finance vs. corporate finance
    \item Special purpose vehicles (SPVs) and ring-fencing
    \item Debt-equity structuring and optimal capital structure
    \item Risk allocation: construction, technology, market, counterparty
    \item Offtake agreements: PPAs, feed-in tariffs, merchant exposure
    \item Sponsor requirements and investment criteria
\end{itemize}

\textit{Session 5.3 - Renewable Energy Financial Modeling (4 hours)}
\begin{itemize}[leftmargin=*]
    \item Revenue modeling: generation profiles, capacity factors, degradation
    \item Cost structure: CAPEX (equipment, installation), OPEX (maintenance, insurance)
    \item Debt modeling: sizing, sculpting, covenants
    \item Returns metrics: project IRR, equity IRR, DSCR, NPV
    \item Sensitivity analysis: key value drivers
    \item Taxation and incentives (ITC, PTC, accelerated depreciation)
\end{itemize}

\textit{Session 5.4 - Project Finance Workshop (2.5 hours)}
\begin{itemize}[leftmargin=*]
    \item Hands-on: Building integrated solar project model in Excel
    \item Case study: Real-world wind farm financing
    \item Risk analysis and mitigation strategies
    \item PPA negotiation dynamics
\end{itemize}

\textbf{Readings:}
\begin{itemize}[leftmargin=*]
    \item IRENA (2024). ``Renewable Power Generation Costs''
    \item Steffen (2020). ``Estimating the cost of capital for renewable energy projects''
    \item Ameli et al. (2021). ``Higher cost of finance exacerbates energy divide''
    \item IEA (2024). ``World Energy Investment Report''
\end{itemize}

\textbf{Midterm Project Due (End of Week 5):} Renewable Energy Project Financial Model

\subsection*{Week 6: Global Regulatory Frameworks (14 hours)}

\textbf{Learning Objectives:}
\begin{itemize}[leftmargin=*]
    \item Navigate EU sustainable finance regulation: Taxonomy, SFDR, CSRD
    \item Understand global regulatory landscape and convergence trends
    \item Apply taxonomies to investment and product classification
    \item Assess regulatory compliance requirements and implementation challenges
    \item Anticipate future regulatory developments
\end{itemize}

\textbf{Session Breakdown:}

\textit{Session 6.1 - EU Taxonomy Deep Dive (4 hours)}
\begin{itemize}[leftmargin=*]
    \item EU Action Plan on Sustainable Finance: origins and objectives
    \item Taxonomy Regulation: structure and six environmental objectives
    \item Technical screening criteria: ``do no significant harm'' and minimum safeguards
    \item Substantial contribution thresholds by sector
    \item Climate Delegated Acts: mitigation and adaptation
    \item Taxonomy alignment reporting for companies and financial products
    \item Controversies and limitations
\end{itemize}

\textit{Session 6.2 - SFDR and Disclosure Requirements (3 hours)}
\begin{itemize}[leftmargin=*]
    \item SFDR Level II: entity and product-level disclosures
    \item Article 6, 8, and 9 fund classifications
    \item Principal Adverse Impacts (PAIs)
    \item Regulatory Technical Standards (RTS)
    \item Greenwashing prevention and enforcement
    \item Integration with MiFID II, UCITS, AIFMD
\end{itemize}

\textit{Session 6.3 - CSRD and Global Reporting Standards (3 hours)}
\begin{itemize}[leftmargin=*]
    \item Corporate Sustainability Reporting Directive (CSRD)
    \item European Sustainability Reporting Standards (ESRS)
    \item Double materiality: impact and financial materiality
    \item ISSB standards: IFRS S1 and S2
    \item GRI, SASB, and standard convergence
    \item Assurance and audit requirements
\end{itemize}

\textit{Session 6.4 - Global Regulatory Landscape (4 hours)}
\begin{itemize}[leftmargin=*]
    \item United States: SEC climate disclosure rule (status and challenges)
    \item United Kingdom: sustainability disclosure requirements
    \item Asia-Pacific: China Green Bond Catalogue, Singapore Green Finance taxonomy, ASEAN standards
    \item Emerging markets: Brazil, India, South Africa
    \item Regulatory convergence and fragmentation
    \item IOSCO and global coordination efforts
    \item Workshop: Taxonomy alignment exercise
\end{itemize}

\textbf{Readings:}
\begin{itemize}[leftmargin=*]
    \item European Commission (2024). ``EU Taxonomy Compass''
    \item Ehlers \& Packer (2017). ``Green bond finance and certification''
    \item Volz (2018). ``Fostering green finance for sustainable development''
    \item ISSB (2023). ``IFRS S1 and S2 Standards''
    \item SEC (2024). ``Climate-Related Disclosures'' (proposed rule)
\end{itemize}

\textbf{Assignment 3 Due (End of Week 6):} Taxonomy Alignment Assessment and Regulatory Compliance Analysis

\subsection*{Week 7: Impact Investing, Blended Finance, and Natural Capital (13 hours)}

\textbf{Learning Objectives:}
\begin{itemize}[leftmargin=*]
    \item Differentiate impact investing from ESG integration and philanthropy
    \item Apply impact measurement frameworks: IRIS+, SDG alignment, Theory of Change
    \item Structure blended finance transactions for emerging markets
    \item Understand natural capital, biodiversity finance, and TNFD framework
    \item Evaluate social bonds and development finance instruments
\end{itemize}

\textbf{Session Breakdown:}

\textit{Session 7.1 - Impact Investing Foundations (3.5 hours)}
\begin{itemize}[leftmargin=*]
    \item Definition and spectrum: financial-first to impact-first
    \item Impact investing vs. ESG vs. philanthropy
    \item Market size: \$1+ trillion AUM globally
    \item Investor motivations and return expectations
    \item Impact themes: financial inclusion, affordable housing, clean energy access, sustainable agriculture
    \item Case studies: microfinance, off-grid solar, impact bonds
\end{itemize}

\textit{Session 7.2 - Impact Measurement and Management (3 hours)}
\begin{itemize}[leftmargin=*]
    \item Theory of Change and impact pathways
    \item IRIS+ metrics: Global Impact Investing Network (GIIN) standards
    \item Impact Management Project (IMP): five dimensions
    \item SDG alignment and contribution claims
    \item Additionality and attribution challenges
    \item SROI (Social Return on Investment) calculations
    \item Reporting frameworks and transparency
\end{itemize}

\textit{Session 7.3 - Blended Finance and Development Capital (3.5 hours)}
\begin{itemize}[leftmargin=*]
    \item Blended finance: concept and rationale
    \item Catalytic capital structures: first-loss, guarantees, concessional debt
    \item DFI roles: IFC, EIB, AfDB, ADB mobilization strategies
    \item Use cases: renewable energy in frontier markets, SME finance, sustainable infrastructure
    \item Performance evidence and leverage ratios
    \item Social bonds: market growth and use of proceeds
    \item Development impact bonds and pay-for-success models
\end{itemize}

\textit{Session 7.4 - Natural Capital and Biodiversity Finance (3 hours)}
\begin{itemize}[leftmargin=*]
    \item Natural capital: definition and valuation approaches
    \item Biodiversity loss as financial risk
    \item TNFD (Taskforce on Nature-related Financial Disclosures) framework
    \item Nature-based solutions and financing mechanisms
    \item Biodiversity credits and markets
    \item Blue bonds and ocean finance
    \item Payment for ecosystem services (PES)
    \item Case study: Conservation finance transactions
\end{itemize}

\textbf{Readings:}
\begin{itemize}[leftmargin=*]
    \item Bugg-Levine \& Emerson (2011). ``Impact Investing: Transforming How We Make Money''
    \item GIIN (2024). ``Annual Impact Investor Survey''
    \item Convergence (2024). ``State of Blended Finance Report''
    \item TNFD (2023). ``Recommendations of the Taskforce on Nature-related Financial Disclosures''
    \item Dasgupta (2021). ``The Economics of Biodiversity'' (Dasgupta Review excerpts)
\end{itemize}

\textbf{Assignment:} None (focus on final project)

\subsection*{Week 8: Advanced Applications, Integration, and Future Trends (14 hours)}

\textbf{Learning Objectives:}
\begin{itemize}[leftmargin=*]
    \item Apply integrated green finance frameworks to complex real-world scenarios
    \item Analyze sovereign green bond strategies and national green finance policies
    \item Understand corporate transition strategies and sector pathways
    \item Explore green fintech innovation and digital solutions
    \item Synthesize course learnings into comprehensive investment strategies
    \item Present professional-quality green finance recommendations
\end{itemize}

\textbf{Session Breakdown:}

\textit{Session 8.1 - Portfolio Construction with Climate Constraints (3 hours)}
\begin{itemize}[leftmargin=*]
    \item Net-zero portfolio alignment strategies
    \item Paris-aligned benchmarks and indices
    \item Climate budget constraints
    \item Transition pathway optimization
    \item Portfolio decarbonization approaches
    \item Balancing returns, risk, and climate objectives
    \item Workshop: Climate-constrained portfolio optimization
\end{itemize}

\textit{Session 8.2 - Sovereign Green Finance and National Strategies (3 hours)}
\begin{itemize}[leftmargin=*]
    \item Sovereign green bonds: major issuers and frameworks
    \item National green finance roadmaps: China, Singapore, UK
    \item Central bank green finance initiatives and NGFS
    \item Green QE and monetary policy tools
    \item Fiscal policy: green budgets and carbon pricing
    \item Case study: EU Green Bond framework (NextGenerationEU)
\end{itemize}

\textit{Session 8.3 - Corporate Transition Strategies and Sector Pathways (3 hours)}
\begin{itemize}[leftmargin=*]
    \item Corporate net-zero commitments and credibility assessment
    \item Science-Based Targets initiative (SBTi)
    \item Sector transition pathways: energy, transport, buildings, heavy industry
    \item Transition finance vs. green finance
    \item Just transition considerations
    \item Engagement strategies for high-emitting sectors
    \item Workshop: Corporate transition plan analysis
\end{itemize}

\textit{Session 8.4 - Green Fintech and Innovation (2 hours)}
\begin{itemize}[leftmargin=*]
    \item Digital MRV (monitoring, reporting, verification)
    \item Blockchain for carbon credits and supply chain transparency
    \item AI and machine learning for climate risk assessment
    \item Satellite data and alternative data sources
    \item Green neobanks and sustainability-linked banking
    \item Crowdfunding and retail green finance platforms
\end{itemize}

\textit{Session 8.5 - Integration Case Study and Course Synthesis (3 hours)}
\begin{itemize}[leftmargin=*]
    \item Comprehensive case study integrating all course elements
    \item Group problem-solving: real-world green finance challenge
    \item Course themes synthesis
    \item Career pathways in green finance
    \item Continuing professional development resources
    \item Final project Q\&A and preparation
\end{itemize}

\textbf{Readings:}
\begin{itemize}[leftmargin=*]
    \item Network for Greening the Financial System (2024). ``Progress Report''
    \item Science Based Targets initiative (2024). ``Corporate Net-Zero Standard''
    \item IEA (2024). ``Net Zero Roadmap''
    \item UN Environment Programme (2024). ``State of Finance for Nature''
    \item Various: Selected readings on fintech and innovation
\end{itemize}

\textbf{Final Project Presentations (Session 8 afternoon):}
\begin{itemize}[leftmargin=*]
    \item Student presentations of final integration projects
    \item Peer and instructor feedback
    \item Best practices sharing
    \item Certificate ceremony
\end{itemize}

\newpage

\section*{Assessment Details}

\subsection*{Weekly Assignments and Case Studies (30\% total)}

\textbf{Assignment 1: Green Bond Framework Evaluation (7.5\%)}
\begin{itemize}[leftmargin=*]
    \item Due: End of Week 2
    \item Format: 2000-2500 words written analysis + Excel pricing model
    \item Task: Select real green bond issuance, evaluate framework against GBP, assess external review quality, analyze pricing and compare with comparable conventional bond
\end{itemize}

\textbf{Assignment 2: ESG Portfolio Construction (7.5\%)}
\begin{itemize}[leftmargin=*]
    \item Due: End of Week 3
    \item Format: Python Jupyter notebook with analysis and commentary
    \item Task: Build multi-criteria ESG-screened portfolio, backtest performance, analyze ESG characteristics, create visualization dashboard, compare with benchmarks
\end{itemize}

\textbf{Assignment 3: Taxonomy Alignment and Regulatory Analysis (7.5\%)}
\begin{itemize}[leftmargin=*]
    \item Due: End of Week 6
    \item Format: 1500-2000 words + Excel alignment worksheet
    \item Task: Assess portfolio or fund for EU Taxonomy alignment, analyze compliance challenges, compare EU and other jurisdictions, recommend disclosure approach
\end{itemize}

\textbf{Assignment 4: Weekly Participation and Mini-Cases (7.5\%)}
\begin{itemize}[leftmargin=*]
    \item Ongoing throughout course
    \item In-class case discussions, workshop exercises, participation quality
\end{itemize}

\subsection*{Midterm Financial Modeling Project (20\%)}

\begin{itemize}[leftmargin=*]
    \item Due: End of Week 5
    \item Format: Complete Excel financial model + 3-page executive summary
    \item Task: Develop comprehensive project finance model for renewable energy project (solar or wind)
    \item Requirements:
    \begin{itemize}
        \item Revenue modeling: generation profile, PPA, degradation
        \item Cost structure: CAPEX, OPEX, working capital
        \item Financing: debt sizing, sculpting, equity returns
        \item Outputs: IRR, NPV, DSCR, payback period
        \item Sensitivity analysis on 5+ key variables
        \item Investment recommendation with risk assessment
    \end{itemize}
\end{itemize}

\subsection*{Research Paper (20\%)}

\begin{itemize}[leftmargin=*]
    \item Due: Week 7 (before final presentations)
    \item Length: 4000-5000 words
    \item Format: Academic research paper with proper citations
    \item Requirements: Minimum 20 peer-reviewed sources, original analysis or application
    \item Sample topics:
    \begin{itemize}
        \item Climate risk integration in pension fund portfolios
        \item Green bond market development in emerging economies
        \item ESG rating methodologies: comparative analysis and implications
        \item Effectiveness of carbon pricing in driving investment decisions
        \item Blended finance: performance analysis and scaling challenges
        \item Natural capital integration in financial decision-making
        \item Regulatory fragmentation vs. convergence in sustainable finance
        \item Custom topic (requires approval by Week 3)
    \end{itemize}
\end{itemize}

\subsection*{Final Integration Project and Presentation (30\% total)}

\textbf{Final Project (20\%)}
\begin{itemize}[leftmargin=*]
    \item Due: End of Week 8 (before presentations)
    \item Format: Professional deliverable (varies by project type) + comprehensive documentation
    \item Options (choose one):
    \begin{itemize}
        \item Green investment strategy: develop complete investment strategy for institutional investor including climate risk assessment, ESG integration, impact goals, and portfolio construction
        \item Corporate sustainability finance plan: design green finance roadmap for company including green bond issuance, transition finance, target setting, and disclosure strategy
        \item Green finance product design: structure new sustainable finance product including framework, verification, pricing, and marketing strategy
        \item Policy analysis and recommendations: analyze national/regional green finance policy framework and provide reform recommendations
    \end{itemize}
    \item Requirements: Integration of minimum 5 course topics, quantitative analysis, practical applicability, professional quality
\end{itemize}

\textbf{Final Presentation (10\%)}
\begin{itemize}[leftmargin=*]
    \item Delivered: Final afternoon of Week 8
    \item Length: 20 minutes + 5 minutes Q\&A
    \item Format: Professional pitch-style presentation (PowerPoint/PDF)
    \item Content: Project overview, methodology, key findings, recommendations, implementation considerations
    \item Evaluation criteria: Technical accuracy (40\%), clarity and structure (30\%), professional delivery (20\%), Q\&A handling (10\%)
\end{itemize}

\newpage

\section*{Comprehensive Reading List}

\subsection*{Core Textbooks and Reviews (Recommended)}
\begin{itemize}[leftmargin=*]
    \item Giglio, S., Kelly, B., and Stroebel, J. (2021). ``Climate Finance.'' \textit{Annual Review of Financial Economics}, 13, 15-36.
    \item Krosinsky, C., and Purdom, S. (Eds.). (2020). \textit{Sustainable Investing: Revolutions in Theory and Practice}. Routledge.
\end{itemize}

\subsection*{Week-by-Week Academic Papers and Reports}

\textbf{Week 1: Foundations}
\begin{itemize}[leftmargin=*]
    \item Berrou, R., Dessertine, P., and Migliorelli, M. (2019). ``An overview of green finance.'' In \textit{The Rise of Green Finance in Europe}.
    \item UNEP (2024). ``Global Landscape of Climate Finance.''
    \item Climate Bonds Initiative (2024). ``State of the Market Report.''
\end{itemize}

\textbf{Week 2: Green Bonds}
\begin{itemize}[leftmargin=*]
    \item ICMA (2024). ``Green Bond Principles: Voluntary Process Guidelines.''
    \item Flammer, C. (2021). ``Corporate green bonds.'' \textit{Journal of Financial Economics}, 142(2), 499-516.
    \item Zerbib, O. D. (2019). ``The effect of pro-environmental preferences on bond prices.'' \textit{Journal of Banking and Finance}, 98, 39-60.
    \item Tang, D. Y., and Zhang, Y. (2020). ``Do shareholders benefit from green bonds?'' \textit{Journal of Corporate Finance}, 61, 101427.
    \item ICMA (2023). ``Climate Transition Finance Handbook.''
\end{itemize}

\textbf{Week 3: ESG}
\begin{itemize}[leftmargin=*]
    \item Berg, F., Kolbel, J. F., and Rigobon, R. (2022). ``Aggregate confusion: The divergence of ESG ratings.'' \textit{Review of Finance}, 26(6), 1315-1344.
    \item Khan, M., Serafeim, G., and Yoon, A. (2016). ``Corporate sustainability: First evidence on materiality.'' \textit{The Accounting Review}, 91(6), 1697-1724.
    \item Eccles, R. G., and Stroehle, J. C. (2018). ``Exploring social origins in the adoption of ESG.''
    \item Dimson, E., Karakas, O., and Li, X. (2015). ``Active ownership.'' \textit{Review of Financial Studies}, 28(12), 3225-3268.
\end{itemize}

\textbf{Week 4: Climate Risk}
\begin{itemize}[leftmargin=*]
    \item TCFD (2017). ``Final Report: Recommendations of the Task Force on Climate-related Financial Disclosures.''
    \item Bolton, P., et al. (2020). \textit{The green swan: Central banking and financial stability in the age of climate change}. Bank for International Settlements.
    \item Battiston, S., et al. (2017). ``A climate stress-test of the financial system.'' \textit{Nature Climate Change}, 7, 283-288.
    \item Krueger, P., Sautner, Z., and Starks, L. T. (2020). ``The importance of climate risks for institutional investors.'' \textit{Review of Financial Studies}, 33(3), 1067-1111.
    \item NGFS (2024). ``NGFS Climate Scenarios for Central Banks and Supervisors.''
\end{itemize}

\textbf{Week 5: Renewable Energy Finance}
\begin{itemize}[leftmargin=*]
    \item IRENA (2024). ``Renewable Power Generation Costs.''
    \item Steffen, B. (2020). ``Estimating the cost of capital for renewable energy projects.'' \textit{Energy Economics}, 88, 104783.
    \item Ameli, N., et al. (2021). ``Higher cost of finance exacerbates a climate investment trap in developing economies.'' \textit{Nature Communications}, 12, 4046.
    \item IEA (2024). ``World Energy Investment Report.''
\end{itemize}

\textbf{Week 6: Regulation}
\begin{itemize}[leftmargin=*]
    \item European Commission (2024). ``EU Taxonomy for Sustainable Activities.''
    \item Ehlers, T., and Packer, F. (2017). ``Green bond finance and certification.'' \textit{BIS Quarterly Review}, September.
    \item Volz, U. (2018). ``Fostering green finance for sustainable development in Asia.'' \textit{ADBI Working Paper Series}, No. 814.
    \item ISSB (2023). ``IFRS Sustainability Disclosure Standards S1 and S2.''
    \item SEC (2024). ``The Enhancement and Standardization of Climate-Related Disclosures'' (Proposed Rule).
\end{itemize}

\textbf{Week 7: Impact and Natural Capital}
\begin{itemize}[leftmargin=*]
    \item Bugg-Levine, A., and Emerson, J. (2011). \textit{Impact Investing: Transforming How We Make Money While Making a Difference}. Jossey-Bass. (Selected chapters)
    \item GIIN (2024). ``Annual Impact Investor Survey.''
    \item Convergence (2024). ``The State of Blended Finance.''
    \item TNFD (2023). ``Recommendations of the Taskforce on Nature-related Financial Disclosures.''
    \item Dasgupta, P. (2021). \textit{The Economics of Biodiversity: The Dasgupta Review}. HM Treasury. (Executive Summary)
\end{itemize}

\textbf{Week 8: Integration and Future}
\begin{itemize}[leftmargin=*]
    \item Network for Greening the Financial System (2024). ``Progress Report.''
    \item Science Based Targets initiative (2024). ``Corporate Net-Zero Standard.''
    \item IEA (2024). ``Net Zero Roadmap: A Global Pathway to Keep the 1.5C Goal in Reach.''
    \item UN Environment Programme (2024). ``State of Finance for Nature.''
\end{itemize}

\subsection*{Additional Recommended Resources}
\begin{itemize}[leftmargin=*]
    \item Journals: Journal of Sustainable Finance \& Investment, Journal of Environmental Economics and Management, Climatic Change, Environmental Research Letters
    \item Industry reports: Climate Bonds Initiative (quarterly), Bloomberg NEF, Rocky Mountain Institute
    \item Databases: MSCI ESG Research, Refinitiv ESG, Sustainalytics, CDP (Carbon Disclosure Project)
    \item Podcasts: Investing in Climate, Green Finance (Yale), Outrage + Optimism
    \item Online courses for continued learning: Coursera Sustainable Finance, edX Climate courses
\end{itemize}

\newpage

\section*{Course Policies}

\subsection*{Attendance and Participation}
\begin{itemize}[leftmargin=*]
    \item Attendance is mandatory for all sessions
    \item Maximum absences: 10\% of contact hours (10.8 hours / approximately one full day)
    \item Exceeding absence limit results in non-certification
    \item Active participation expected in workshops, discussions, and group exercises
    \item Cameras on for any remote components; professional environment required
\end{itemize}

\subsection*{Academic Integrity}
\begin{itemize}[leftmargin=*]
    \item All work must be original and properly attributed
    \item Proper citation required for all sources (APA or Chicago format)
    \item Collaboration permitted on workshop exercises but not on individual assignments
    \item Use of AI tools (ChatGPT, etc.) must be disclosed and properly cited
    \item Violations result in failing grade and possible program dismissal
    \item Honor code: signed acknowledgment required at program start
\end{itemize}

\subsection*{Late Submissions}
\begin{itemize}[leftmargin=*]
    \item Late penalty: 10\% per day (24-hour periods from deadline)
    \item Maximum late acceptance: 3 days
    \item After 3 days: zero grade unless extenuating circumstances with documentation
    \item Extensions available for documented medical/emergency reasons (request in advance when possible)
    \item Final project and presentation: no late submissions accepted
\end{itemize}

\subsection*{Grading Scale}
\begin{itemize}[leftmargin=*]
    \item 90-100\%: Distinction (awarded on certificate)
    \item 80-89\%: Merit (awarded on certificate)
    \item 70-79\%: Pass (certificate awarded)
    \item Below 70\%: Fail (no certificate; may retake specific assessments)
    \item Minimum 70\% required for certificate
    \item Distinction students eligible for recommendation letters
\end{itemize}

\subsection*{Technical Requirements and Support}
\begin{itemize}[leftmargin=*]
    \item Laptop required for all sessions (Windows or Mac)
    \item Software pre-installation: Excel, Python (Anaconda distribution), PDF reader
    \item ESG platform access: provided via institutional licenses (training in Week 3)
    \item IT support: available via email and during breaks
    \item Technical setup session: optional pre-program session offered
\end{itemize}

\subsection*{Feedback and Assessment}
\begin{itemize}[leftmargin=*]
    \item Assignments returned with detailed feedback within 10 business days
    \item Midterm project: feedback session available
    \item Office hours: weekly, 2 hours (schedule provided)
    \item Anonymous mid-program survey for continuous improvement
    \item Final program evaluation required for certificate
\end{itemize}

\subsection*{Diversity, Equity, and Inclusion}
\begin{itemize}[leftmargin=*]
    \item Commitment to inclusive learning environment
    \item Accommodations available for documented disabilities
    \item Respectful dialogue expected; diverse perspectives valued
    \item Materials present global perspectives and diverse case studies
    \item Gender-inclusive language used throughout
\end{itemize}

\subsection*{Networking and Career Support}
\begin{itemize}[leftmargin=*]
    \item Access to alumni network of green finance professionals
    \item Optional career coaching session (1 hour)
    \item LinkedIn group for cohort connection
    \item Job board access for green finance opportunities
    \item Guest speaker networking (when applicable)
\end{itemize}

\vspace{1cm}

\section*{Program Fees and Logistics}

\subsection*{Tuition}
\begin{itemize}[leftmargin=*]
    \item Program fee: [To be determined based on institutional pricing]
    \item Suggested range: \$5,500 - \$6,900 (competitive with Harvard, ICMA)
    \item Includes: All instruction, materials, ESG platform access, certificate
    \item Does not include: Accommodation, meals, travel, textbooks (readings provided)
    \item Payment plans available; early bird discount (register 8+ weeks in advance)
\end{itemize}

\subsection*{Instructor Information}
\textit{[To be completed with instructor credentials and contact information]}

\vspace{0.5cm}
\hrule
\vspace{0.3cm}
\textit{Note: This syllabus is subject to modification with advance notice. Any changes will be communicated via email and updated syllabus posted to learning management system.}

\vspace{0.5cm}
\textit{Version: 8-Week Intensive | Last updated: November 2025}

\end{document}
