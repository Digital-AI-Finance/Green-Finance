\documentclass[8pt]{article}
\usepackage[margin=1in]{geometry}
\usepackage{enumitem}
\usepackage{hyperref}
\usepackage{booktabs}
\usepackage{longtable}

\title{\textbf{Green Finance Microcredential} \\ Professional Certificate Program}
\author{}
\date{}

\begin{document}

\maketitle

\section*{Course Overview}

\subsection*{Program Description}
This intensive microcredential provides finance professionals with comprehensive knowledge and practical skills in green finance, sustainable investment, and climate risk assessment. The program covers green bonds, ESG integration, climate risk frameworks, renewable energy project finance, and global regulatory standards.

\subsection*{Program Details}
\begin{itemize}[leftmargin=*]
    \item \textbf{Duration:} 6 weeks intensive (80 contact hours)
    \item \textbf{Format:} In-person instruction with hands-on workshops
    \item \textbf{Schedule:} 13-14 hours per week
    \item \textbf{Target Audience:} Finance professionals transitioning to sustainable finance
    \item \textbf{Prerequisites:} None (beginner-friendly with comprehensive foundation building)
    \item \textbf{Credential:} University-issued Professional Certificate in Green Finance
\end{itemize}

\subsection*{Learning Objectives}
Upon completion, participants will be able to:
\begin{enumerate}[leftmargin=*]
    \item Understand core concepts, instruments, and markets in green finance
    \item Analyze and apply ESG data for investment screening and portfolio construction
    \item Assess climate-related financial risks using TCFD and scenario analysis frameworks
    \item Develop financial models for renewable energy projects and green investments
    \item Navigate global regulatory frameworks including EU Taxonomy, SFDR, and emerging standards
    \item Evaluate green bonds and sustainable debt instruments
    \item Apply practical tools (Python, Excel, ESG platforms) for green finance analysis
\end{enumerate}

\subsection*{Assessment Components}
\begin{itemize}[leftmargin=*]
    \item Case Study Analyses (3 assignments): 30\%
    \item Financial Modeling Project: 25\%
    \item Research Paper: 25\%
    \item Final Presentation: 20\%
\end{itemize}

\subsection*{Required Tools and Software}
\begin{itemize}[leftmargin=*]
    \item Microsoft Excel (financial modeling)
    \item Python 3.x with pandas, numpy, matplotlib libraries
    \item Access to ESG data platforms (MSCI, Refinitiv, or Bloomberg ESG - institutional access provided)
    \item PDF reader for course materials
\end{itemize}

\newpage

\section*{Weekly Schedule}

\subsection*{Week 1: Green Finance Fundamentals (14 hours)}

\textbf{Learning Objectives:}
\begin{itemize}[leftmargin=*]
    \item Define green finance and understand its evolution
    \item Identify key stakeholders and market participants
    \item Understand the business case for sustainable finance
    \item Recognize different types of green financial instruments
\end{itemize}

\textbf{Session Breakdown:}

\textit{Session 1.1 - Introduction to Green Finance (3 hours)}
\begin{itemize}[leftmargin=*]
    \item Definition and scope of green finance
    \item Historical development and market growth
    \item Climate change and the financial system
    \item Sustainable Development Goals (SDGs) and finance
\end{itemize}

\textit{Session 1.2 - Green Finance Ecosystem (3 hours)}
\begin{itemize}[leftmargin=*]
    \item Market participants: issuers, investors, intermediaries
    \item Role of development finance institutions
    \item International organizations and standard-setters
    \item Public vs. private green finance
\end{itemize}

\textit{Session 1.3 - Green Financial Instruments Overview (4 hours)}
\begin{itemize}[leftmargin=*]
    \item Green bonds and loans
    \item Sustainability-linked instruments
    \item Green equity and funds
    \item Carbon markets and credits
    \item Blended finance structures
\end{itemize}

\textit{Session 1.4 - Workshop: Financial Fundamentals Review (4 hours)}
\begin{itemize}[leftmargin=*]
    \item Time value of money and discounting
    \item Bond pricing and yield calculations
    \item Portfolio theory basics
    \item Risk and return metrics
    \item Excel modeling foundations
\end{itemize}

\textbf{Readings:}
\begin{itemize}[leftmargin=*]
    \item Berrou et al. (2019). ``A taxonomy of green finance''
    \item Climate Bonds Initiative (2023). ``Green Bond Market Summary''
    \item UNEP (2023). ``Global Landscape of Climate Finance''
\end{itemize}

\textbf{Assignment:} None (foundation week)

\subsection*{Week 2: Green Bonds and Sustainable Debt Instruments (14 hours)}

\textbf{Learning Objectives:}
\begin{itemize}[leftmargin=*]
    \item Understand green bond market structure and evolution
    \item Analyze green bond frameworks and use of proceeds
    \item Evaluate green bond pricing and performance
    \item Apply green bond principles and verification processes
\end{itemize}

\textbf{Session Breakdown:}

\textit{Session 2.1 - Green Bond Markets (3 hours)}
\begin{itemize}[leftmargin=*]
    \item Green bond market history and growth
    \item Market segmentation: sovereign, corporate, municipal
    \item Green Bond Principles (ICMA)
    \item Use of proceeds categories
\end{itemize}

\textit{Session 2.2 - Green Bond Structuring and Issuance (4 hours)}
\begin{itemize}[leftmargin=*]
    \item Framework development
    \item External reviews and verification
    \item Second-party opinions
    \item Reporting and transparency requirements
    \item Case study: Major green bond issuance
\end{itemize}

\textit{Session 2.3 - Green Bond Pricing and Performance (3 hours)}
\begin{itemize}[leftmargin=*]
    \item Greenium: myth or reality?
    \item Pricing analysis methodologies
    \item Performance comparison with conventional bonds
    \item Liquidity considerations
\end{itemize}

\textit{Session 2.4 - Workshop: Green Bond Analysis (4 hours)}
\begin{itemize}[leftmargin=*]
    \item Hands-on: Analyzing green bond frameworks
    \item Excel exercise: Green bond pricing
    \item Group activity: Framework critique
\end{itemize}

\textbf{Readings:}
\begin{itemize}[leftmargin=*]
    \item ICMA (2023). ``Green Bond Principles''
    \item Flammer (2021). ``Corporate green bonds'' (Journal of Financial Economics)
    \item Zerbib (2019). ``The effect of pro-environmental preferences on bond prices''
\end{itemize}

\textbf{Assignment 1 Due (End of Week 2):} Case Study Analysis - Green Bond Framework Evaluation

\subsection*{Week 3: ESG Integration and Data Analysis (13 hours)}

\textbf{Learning Objectives:}
\begin{itemize}[leftmargin=*]
    \item Understand ESG rating methodologies and providers
    \item Apply ESG screening and integration techniques
    \item Use Python and ESG platforms for data analysis
    \item Evaluate materiality and ESG performance metrics
\end{itemize}

\textbf{Session Breakdown:}

\textit{Session 3.1 - ESG Frameworks and Ratings (3 hours)}
\begin{itemize}[leftmargin=*]
    \item ESG concepts and evolution
    \item Major ESG rating providers (MSCI, Sustainalytics, Refinitiv)
    \item Rating methodologies and controversies
    \item ESG data quality and comparability issues
\end{itemize}

\textit{Session 3.2 - ESG Integration Strategies (3 hours)}
\begin{itemize}[leftmargin=*]
    \item Negative screening vs. positive screening
    \item ESG integration approaches
    \item Thematic investing
    \item Impact investing vs. ESG investing
    \item Materiality assessment
\end{itemize}

\textit{Session 3.3 - Workshop: ESG Data Platforms (3 hours)}
\begin{itemize}[leftmargin=*]
    \item Hands-on: MSCI/Refinitiv platform navigation
    \item Data extraction and export
    \item Company ESG profiles analysis
    \item Sector comparison tools
\end{itemize}

\textit{Session 3.4 - Workshop: Python for ESG Analysis (4 hours)}
\begin{itemize}[leftmargin=*]
    \item Python basics review
    \item pandas for ESG data manipulation
    \item Portfolio screening with Python
    \item Visualization with matplotlib
    \item Creating ESG scorecards
\end{itemize}

\textbf{Readings:}
\begin{itemize}[leftmargin=*]
    \item Berg et al. (2022). ``Aggregate confusion: ESG ratings''
    \item Eccles and Stroehle (2018). ``Exploring social origins in ESG''
    \item Khan et al. (2016). ``Corporate sustainability: First evidence on materiality''
\end{itemize}

\textbf{Assignment 2 Due (End of Week 3):} Python ESG Portfolio Analysis Project

\subsection*{Week 4: Climate Risk Assessment and TCFD (13 hours)}

\textbf{Learning Objectives:}
\begin{itemize}[leftmargin=*]
    \item Understand physical and transition climate risks
    \item Apply TCFD framework for climate risk disclosure
    \item Conduct scenario analysis for climate risks
    \item Quantify climate-related financial impacts
\end{itemize}

\textbf{Session Breakdown:}

\textit{Session 4.1 - Climate Risk Fundamentals (3 hours)}
\begin{itemize}[leftmargin=*]
    \item Physical risks: acute and chronic
    \item Transition risks: policy, technology, market, reputation
    \item Climate risk transmission channels
    \item Systemic risk considerations
\end{itemize}

\textit{Session 4.2 - TCFD Framework (4 hours)}
\begin{itemize}[leftmargin=*]
    \item TCFD structure: governance, strategy, risk management, metrics
    \item Climate scenario analysis
    \item NGFS scenarios and applications
    \item Disclosure best practices
    \item Case study: Corporate TCFD report analysis
\end{itemize}

\textit{Session 4.3 - Climate Risk Quantification (3 hours)}
\begin{itemize}[leftmargin=*]
    \item Climate Value-at-Risk (CVaR)
    \item Carbon footprinting methodologies
    \item Stranded asset analysis
    \item Physical risk modeling approaches
\end{itemize}

\textit{Session 4.4 - Workshop: Scenario Analysis Exercise (3 hours)}
\begin{itemize}[leftmargin=*]
    \item Hands-on: Portfolio climate risk assessment
    \item Excel-based scenario modeling
    \item Group exercise: Sector-specific risk analysis
\end{itemize}

\textbf{Readings:}
\begin{itemize}[leftmargin=*]
    \item TCFD (2017). ``Final Report: Recommendations''
    \item Bolton et al. (2020). ``The green swan'' (BIS)
    \item Battiston et al. (2017). ``A climate stress-test of the financial system''
\end{itemize}

\textbf{Assignment:} None (focus on midpoint integration)

\subsection*{Week 5: Renewable Energy Project Finance (13 hours)}

\textbf{Learning Objectives:}
\begin{itemize}[leftmargin=*]
    \item Understand renewable energy technologies and economics
    \item Develop financial models for solar and wind projects
    \item Evaluate project risks and mitigation strategies
    \item Analyze power purchase agreements (PPAs)
\end{itemize}

\textbf{Session Breakdown:}

\textit{Session 5.1 - Renewable Energy Technologies (3 hours)}
\begin{itemize}[leftmargin=*]
    \item Solar PV: technology and cost trends
    \item Wind energy: onshore and offshore
    \item Energy storage and grid integration
    \item Emerging technologies: green hydrogen, geothermal
\end{itemize}

\textit{Session 5.2 - Project Finance Fundamentals (3 hours)}
\begin{itemize}[leftmargin=*]
    \item Project finance structure and rationale
    \item Special purpose vehicles (SPVs)
    \item Debt-equity structures
    \item Risk allocation in project finance
    \item Offtake agreements and revenue certainty
\end{itemize}

\textit{Session 5.3 - Renewable Energy Financial Modeling (4 hours)}
\begin{itemize}[leftmargin=*]
    \item Revenue modeling: capacity factors, degradation
    \item Cost structure: CAPEX, OPEX
    \item Debt sizing and sculpting
    \item Returns metrics: IRR, DSCR, NPV
    \item Sensitivity and scenario analysis
\end{itemize}

\textit{Session 5.4 - Workshop: Solar Project Model (3 hours)}
\begin{itemize}[leftmargin=*]
    \item Hands-on: Building Excel model for solar project
    \item Case study: Real-world solar project evaluation
    \item Group work: Risk analysis
\end{itemize}

\textbf{Readings:}
\begin{itemize}[leftmargin=*]
    \item IRENA (2023). ``Renewable Power Generation Costs''
    \item Steffen (2020). ``Estimating the cost of capital for renewable energy projects''
    \item Ameli et al. (2021). ``Higher cost of finance exacerbates energy divide''
\end{itemize}

\textbf{Assignment 3 Due (End of Week 5):} Renewable Energy Project Financial Model

\subsection*{Week 6: Regulatory Frameworks and Course Integration (13 hours)}

\textbf{Learning Objectives:}
\begin{itemize}[leftmargin=*]
    \item Navigate EU Taxonomy and SFDR requirements
    \item Understand global regulatory landscape
    \item Apply integrated green finance analysis
    \item Present comprehensive green finance strategies
\end{itemize}

\textbf{Session Breakdown:}

\textit{Session 6.1 - EU Green Finance Regulation (4 hours)}
\begin{itemize}[leftmargin=*]
    \item EU Taxonomy: structure and criteria
    \item Sustainable Finance Disclosure Regulation (SFDR)
    \item Corporate Sustainability Reporting Directive (CSRD)
    \item Compliance challenges and implementation
\end{itemize}

\textit{Session 6.2 - Global Regulatory Landscape (3 hours)}
\begin{itemize}[leftmargin=*]
    \item US: SEC climate disclosure rules
    \item Asia: green taxonomies in China, Singapore, ASEAN
    \item Emerging markets: national green finance policies
    \item International coordination and convergence
\end{itemize}

\textit{Session 6.3 - Integration Workshop (3 hours)}
\begin{itemize}[leftmargin=*]
    \item Cross-cutting themes review
    \item Integrated case study
    \item Career pathways in green finance
    \item Final project Q\&A
\end{itemize}

\textit{Session 6.4 - Final Presentations (3 hours)}
\begin{itemize}[leftmargin=*]
    \item Student presentations of final projects
    \item Peer feedback and discussion
    \item Course wrap-up and certification
\end{itemize}

\textbf{Readings:}
\begin{itemize}[leftmargin=*]
    \item European Commission (2023). ``EU Taxonomy Compass''
    \item Ehlers and Packer (2017). ``Green bond finance and certification''
    \item Volz (2018). ``Fostering green finance for sustainable development''
\end{itemize}

\textbf{Final Deliverables Due:}
\begin{itemize}[leftmargin=*]
    \item Research Paper: Due end of Week 6
    \item Final Presentation: Delivered in Session 6.4
\end{itemize}

\newpage

\section*{Assessment Details}

\subsection*{Case Study Analyses (30\% total)}

\textbf{Assignment 1: Green Bond Framework Evaluation (10\%)}
\begin{itemize}[leftmargin=*]
    \item Due: End of Week 2
    \item Length: 1500-2000 words
    \item Task: Analyze a real green bond issuance, evaluate framework against Green Bond Principles, assess use of proceeds alignment, review external verification
\end{itemize}

\textbf{Assignment 2: ESG Portfolio Analysis (10\%)}
\begin{itemize}[leftmargin=*]
    \item Due: End of Week 3
    \item Format: Python Jupyter notebook with analysis
    \item Task: Build ESG-screened portfolio, compare performance with benchmark, analyze ESG score distributions, create visualization dashboard
\end{itemize}

\textbf{Assignment 3: Renewable Energy Project Model (10\%)}
\begin{itemize}[leftmargin=*]
    \item Due: End of Week 5
    \item Format: Excel financial model with 2-page executive summary
    \item Task: Develop complete project finance model for solar/wind project, including sensitivity analysis and investment recommendation
\end{itemize}

\subsection*{Research Paper (25\%)}

\begin{itemize}[leftmargin=*]
    \item Due: End of Week 6
    \item Length: 3500-4500 words
    \item Topic: Choose from provided list or propose custom topic
    \item Requirements: Academic rigor, minimum 15 peer-reviewed sources, original analysis
    \item Sample topics:
    \begin{itemize}
        \item Climate risk integration in institutional portfolios
        \item Green bond market development in emerging economies
        \item Comparative analysis of ESG rating methodologies
        \item Effectiveness of carbon pricing mechanisms
        \item Blended finance structures for renewable energy
    \end{itemize}
\end{itemize}

\subsection*{Final Project and Presentation (45\% total)}

\textbf{Financial Modeling Project (25\%)}
\begin{itemize}[leftmargin=*]
    \item Due: End of Week 6
    \item Format: Comprehensive Excel/Python model with documentation
    \item Options:
    \begin{itemize}
        \item Green bond issuance structuring and pricing
        \item Climate-aware portfolio optimization
        \item Renewable energy investment evaluation
        \item ESG-integrated corporate valuation
    \end{itemize}
    \item Requirements: Complete model, assumptions documentation, sensitivity analysis, executive summary
\end{itemize}

\textbf{Final Presentation (20\%)}
\begin{itemize}[leftmargin=*]
    \item Delivered: Session 6.4
    \item Length: 15 minutes + 5 minutes Q\&A
    \item Content: Present financial modeling project findings, investment recommendation or strategy proposal
    \item Format: Professional pitch-style presentation
    \item Evaluation: Technical accuracy, clarity, practical applicability, presentation skills
\end{itemize}

\newpage

\section*{Required Reading List}

\subsection*{Core Textbooks (Recommended)}
\begin{itemize}[leftmargin=*]
    \item Baker, M., Bergstresser, D., Serafeim, G., and Wurgler, J. (2022). \textit{The Pricing and Ownership of U.S. Green Bonds}. Annual Review of Financial Economics.
    \item Giglio, S., Kelly, B., and Stroebel, J. (2021). \textit{Climate Finance}. Annual Review of Financial Economics.
\end{itemize}

\subsection*{Week-by-Week Academic Papers}

\textbf{Week 1 - Foundations}
\begin{itemize}[leftmargin=*]
    \item Berrou, R., Dessertine, P., and Migliorelli, M. (2019). ``An overview of green finance.'' \textit{The Rise of Green Finance in Europe}.
    \item Sachs, J. D., et al. (2019). ``Six transformations to achieve the SDGs.'' \textit{Nature Sustainability}.
\end{itemize}

\textbf{Week 2 - Green Bonds}
\begin{itemize}[leftmargin=*]
    \item Flammer, C. (2021). ``Corporate green bonds.'' \textit{Journal of Financial Economics}, 142(2), 499-516.
    \item Zerbib, O. D. (2019). ``The effect of pro-environmental preferences on bond prices.'' \textit{Journal of Banking and Finance}, 98, 39-60.
    \item Tang, D. Y., and Zhang, Y. (2020). ``Do shareholders benefit from green bonds?'' \textit{Journal of Corporate Finance}, 61, 101427.
\end{itemize}

\textbf{Week 3 - ESG}
\begin{itemize}[leftmargin=*]
    \item Berg, F., Kolbel, J. F., and Rigobon, R. (2022). ``Aggregate confusion: The divergence of ESG ratings.'' \textit{Review of Finance}, 26(6), 1315-1344.
    \item Khan, M., Serafeim, G., and Yoon, A. (2016). ``Corporate sustainability: First evidence on materiality.'' \textit{The Accounting Review}, 91(6), 1697-1724.
    \item Dimson, E., Karakas, O., and Li, X. (2015). ``Active ownership.'' \textit{Review of Financial Studies}, 28(12), 3225-3268.
\end{itemize}

\textbf{Week 4 - Climate Risk}
\begin{itemize}[leftmargin=*]
    \item Bolton, P., et al. (2020). \textit{The green swan: Central banking and financial stability in the age of climate change}. BIS.
    \item Battiston, S., et al. (2017). ``A climate stress-test of the financial system.'' \textit{Nature Climate Change}, 7, 283-288.
    \item Krueger, P., Sautner, Z., and Starks, L. T. (2020). ``The importance of climate risks for institutional investors.'' \textit{Review of Financial Studies}, 33(3), 1067-1111.
\end{itemize}

\textbf{Week 5 - Project Finance}
\begin{itemize}[leftmargin=*]
    \item Steffen, B. (2020). ``Estimating the cost of capital for renewable energy projects.'' \textit{Energy Economics}, 88, 104783.
    \item Ameli, N., et al. (2021). ``Higher cost of finance exacerbates a climate investment trap in developing economies.'' \textit{Nature Communications}, 12, 4046.
\end{itemize}

\textbf{Week 6 - Regulation}
\begin{itemize}[leftmargin=*]
    \item Ehlers, T., and Packer, F. (2017). ``Green bond finance and certification.'' \textit{BIS Quarterly Review}.
    \item Volz, U. (2018). ``Fostering green finance for sustainable development in Asia.'' \textit{ADBI Working Paper Series}.
\end{itemize}

\subsection*{Industry Reports and Guidelines}
\begin{itemize}[leftmargin=*]
    \item Climate Bonds Initiative. (2023). ``Green Bond Market Summary.''
    \item ICMA. (2023). ``Green Bond Principles: Voluntary Process Guidelines.''
    \item TCFD. (2017). ``Final Report: Recommendations of the Task Force on Climate-related Financial Disclosures.''
    \item UNEP. (2023). ``Global Landscape of Climate Finance.''
    \item IRENA. (2023). ``Renewable Power Generation Costs.''
    \item European Commission. (2023). ``EU Taxonomy for Sustainable Activities.''
    \item NGFS. (2023). ``NGFS Climate Scenarios for Central Banks and Supervisors.''
\end{itemize}

\newpage

\section*{Course Policies}

\subsection*{Attendance}
Attendance is mandatory for all sessions. Participants missing more than 10\% of contact hours (8 hours) will not receive certification.

\subsection*{Academic Integrity}
All submitted work must be original. Proper citation is required for all external sources. Violations will result in failing grade and possible program dismissal.

\subsection*{Late Submissions}
Late assignments will be penalized 10\% per day, up to 3 days. After 3 days, submissions will not be accepted without documented extenuating circumstances.

\subsection*{Grading Scale}
\begin{itemize}[leftmargin=*]
    \item 90-100\%: Distinction
    \item 80-89\%: Merit
    \item 70-79\%: Pass
    \item Below 70\%: Fail (no certificate awarded)
\end{itemize}

\subsection*{Technical Support}
Support for ESG platform access and software issues will be available via email and office hours.

\subsection*{Feedback}
Assignments will be returned with feedback within 10 business days of submission. Students are encouraged to attend office hours for detailed discussions.

\vspace{1cm}

\section*{Instructor Contact Information}
\textit{[To be completed with instructor details]}

\vspace{0.5cm}
\hrule
\vspace{0.3cm}
\textit{Note: This syllabus is subject to modification with advance notice. Updates will be communicated via email.}

\end{document}
