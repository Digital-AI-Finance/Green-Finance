\documentclass[8pt]{beamer}
\usetheme{Madrid}
\usecolortheme{default}
\usepackage{graphicx}
\usepackage{booktabs}
\usepackage{tikz}
\usepackage{hyperref}

\title{Week 1: Green Finance Fundamentals}
\subtitle{Professional Certificate in Green Finance}
\author{}
\date{}

\begin{document}

\begin{frame}
\titlepage
\end{frame}

\begin{frame}{Outline}
\tableofcontents
\end{frame}

\section{Introduction to Green Finance}

\begin{frame}{What is Green Finance?}
\begin{columns}[T]
\column{0.48\textwidth}
\textbf{Definition}
\begin{itemize}
    \item Financial investments flowing to sustainable development
    \item Support for environmental and climate objectives
    \item Integration of environmental considerations into financial decisions
\end{itemize}

\vspace{1em}
\textbf{Key Characteristics}
\begin{itemize}
    \item Environmental additionality
    \item Measurable impact
    \item Transparency and reporting
    \item Risk-adjusted returns
\end{itemize}

\column{0.48\textwidth}
\textbf{Scope}
\begin{itemize}
    \item Climate change mitigation
    \item Climate change adaptation
    \item Pollution prevention
    \item Biodiversity conservation
    \item Sustainable resource use
    \item Circular economy
\end{itemize}

\vspace{1em}
\textbf{Market Size}
\begin{itemize}
    \item Global green finance: \$5+ trillion annually
    \item Green bond market: \$1.6 trillion outstanding (2023)
    \item Growing 30-40\% per year
\end{itemize}
\end{columns}
\end{frame}

\begin{frame}{Why Green Finance Matters}
\begin{columns}[T]
\column{0.48\textwidth}
\textbf{Climate Imperative}
\begin{itemize}
    \item Paris Agreement: limit warming to 1.5-2C
    \item Requires \$3-5 trillion annual investment
    \item Current investment gap: \$2-3 trillion/year
    \item Financial system must mobilize capital
\end{itemize}

\vspace{1em}
\textbf{Financial Risk}
\begin{itemize}
    \item Physical risks: extreme weather, sea level rise
    \item Transition risks: policy, technology, markets
    \item Stranded assets: fossil fuel reserves
    \item Systemic risk to financial stability
\end{itemize}

\column{0.48\textwidth}
\textbf{Business Opportunity}
\begin{itemize}
    \item Clean energy: \$10+ trillion market
    \item Green infrastructure investment needs
    \item Innovation in sustainable technologies
    \item First-mover advantages
\end{itemize}

\vspace{1em}
\textbf{Regulatory Drivers}
\begin{itemize}
    \item EU Taxonomy and SFDR
    \item SEC climate disclosure rules
    \item Central bank climate stress tests
    \item Mandatory TCFD reporting
\end{itemize}
\end{columns}
\end{frame}

\begin{frame}{Historical Development}
\begin{columns}[T]
\column{0.48\textwidth}
\textbf{Early History (Pre-2007)}
\begin{itemize}
    \item 1990s: Socially responsible investing (SRI)
    \item 2000: UN Global Compact
    \item 2006: UN Principles for Responsible Investment
    \item Focus: ethical exclusion and advocacy
\end{itemize}

\vspace{1em}
\textbf{Emergence (2007-2015)}
\begin{itemize}
    \item 2007: First green bond (EIB)
    \item 2008: Financial crisis spurs sustainable finance
    \item 2014: Green Bond Principles launched
    \item 2015: Paris Agreement catalyzes growth
\end{itemize}

\column{0.48\textwidth}
\textbf{Mainstreaming (2015-2020)}
\begin{itemize}
    \item 2017: TCFD recommendations
    \item Explosive green bond growth
    \item ESG integration accelerates
    \item Central banks engage climate risk
\end{itemize}

\vspace{1em}
\textbf{Maturation (2020-Present)}
\begin{itemize}
    \item 2021: EU Taxonomy implemented
    \item 2022: SFDR disclosure requirements
    \item 2023: SEC climate rule proposals
    \item Regulatory frameworks solidify
    \item Focus on impact and greenwashing prevention
\end{itemize}
\end{columns}
\end{frame}

\begin{frame}{Green Finance vs. Traditional Finance}
\begin{table}
\centering
\small
\begin{tabular}{lll}
\toprule
\textbf{Dimension} & \textbf{Traditional Finance} & \textbf{Green Finance} \\
\midrule
Primary objective & Financial returns & Returns + environmental impact \\
Risk assessment & Financial risks only & Financial + climate risks \\
Time horizon & Quarterly/annual & Long-term (decades) \\
Externalities & Ignored/unpriced & Internalized/measured \\
Reporting & Financial statements & Financial + non-financial \\
Valuation & NPV, IRR & NPV + impact metrics \\
Regulation & Prudential only & Prudential + sustainability \\
Transparency & Limited & Enhanced disclosure \\
\bottomrule
\end{tabular}
\end{table}

\vspace{1em}
\small
\textbf{Key Insight:} Green finance does not trade-off returns for impact. Studies show comparable or superior risk-adjusted returns with better long-term resilience.
\end{frame}

\section{Green Finance Ecosystem}

\begin{frame}{Market Participants}
\begin{columns}[T]
\column{0.48\textwidth}
\textbf{Capital Providers}
\begin{itemize}
    \item Institutional investors (pensions, insurance)
    \item Asset managers and funds
    \item Banks and commercial lenders
    \item Development finance institutions
    \item Retail investors
    \item Corporations (internal capital)
\end{itemize}

\vspace{1em}
\textbf{Capital Recipients}
\begin{itemize}
    \item Governments (sovereign bonds)
    \item Corporations (green bonds, loans)
    \item Project developers (renewable energy)
    \item Financial institutions (sustainability-linked)
    \item Municipalities (green municipal bonds)
\end{itemize}

\column{0.48\textwidth}
\textbf{Intermediaries}
\begin{itemize}
    \item Investment banks (underwriting)
    \item Commercial banks (lending)
    \item Stock exchanges (listing)
    \item Rating agencies
    \item Verifiers and certifiers
\end{itemize}

\vspace{1em}
\textbf{Standard-Setters and Regulators}
\begin{itemize}
    \item ICMA (Green Bond Principles)
    \item Climate Bonds Initiative
    \item EU Commission (Taxonomy)
    \item IOSCO, FSB (global coordination)
    \item Central banks and regulators
    \item ISSB (sustainability standards)
\end{itemize}
\end{columns}
\end{frame}

\begin{frame}{Role of Development Finance Institutions}
\begin{columns}[T]
\column{0.48\textwidth}
\textbf{Major DFIs}
\begin{itemize}
    \item World Bank/IFC
    \item European Investment Bank (EIB)
    \item Asian Development Bank (ADB)
    \item African Development Bank (AfDB)
    \item Green Climate Fund (GCF)
    \item National DFIs (KfW, AFD, JICA)
\end{itemize}

\vspace{1em}
\textbf{Functions}
\begin{itemize}
    \item Catalyze private capital
    \item Market creation and demonstration
    \item Technical assistance
    \item Risk mitigation (guarantees)
\end{itemize}

\column{0.48\textwidth}
\textbf{Blended Finance Model}
\begin{itemize}
    \item Concessional capital: grants, soft loans
    \item Commercial capital: private investment
    \item Risk layering and tranching
    \item First-loss protection
    \item Particularly important in emerging markets
\end{itemize}

\vspace{1em}
\textbf{Impact}
\begin{itemize}
    \item EIB: 200+ billion EUR climate finance (2015-2023)
    \item IFC: 15 billion USD green bonds issued
    \item GCF: 13 billion USD committed
    \item Leverage ratios: 1:3 to 1:10
\end{itemize}
\end{columns}
\end{frame}

\begin{frame}{International Organizations and Standard-Setters}
\begin{columns}[T]
\column{0.48\textwidth}
\textbf{UN System}
\begin{itemize}
    \item UNEP Finance Initiative
    \item UNFCCC (Paris Agreement)
    \item UN Sustainable Development Goals
    \item Principles for Responsible Investment (PRI)
\end{itemize}

\vspace{1em}
\textbf{Financial Stability}
\begin{itemize}
    \item Financial Stability Board (FSB)
    \item Network for Greening the Financial System (NGFS)
    \item Bank for International Settlements
    \item International Monetary Fund
\end{itemize}

\column{0.48\textwidth}
\textbf{Market Standards}
\begin{itemize}
    \item ICMA: Green Bond Principles
    \item Climate Bonds Initiative: Certification
    \item ISSB: Sustainability standards (IFRS S1/S2)
    \item GRI: Sustainability reporting
\end{itemize}

\vspace{1em}
\textbf{Regional Bodies}
\begin{itemize}
    \item EU: Taxonomy, SFDR, CSRD
    \item ASEAN: Green Bond Standards
    \item China: Green Bond Catalogue
    \item Various national regulators
\end{itemize}
\end{columns}
\end{frame}

\section{Green Financial Instruments}

\begin{frame}{Overview of Green Instruments}
\begin{columns}[T]
\column{0.48\textwidth}
\textbf{Debt Instruments}
\begin{itemize}
    \item Green bonds
    \item Green loans
    \item Sustainability-linked bonds
    \item Sustainability-linked loans
    \item Transition bonds
    \item Blue bonds (ocean finance)
\end{itemize}

\vspace{1em}
\textbf{Equity Instruments}
\begin{itemize}
    \item Green stocks/IPOs
    \item Green funds and ETFs
    \item Private equity in clean tech
    \item Venture capital for climate tech
\end{itemize}

\column{0.48\textwidth}
\textbf{Other Instruments}
\begin{itemize}
    \item Carbon credits and offsets
    \item Green securitization
    \item Catastrophe bonds
    \item Green derivatives
    \item Payment for ecosystem services
\end{itemize}

\vspace{1em}
\textbf{Market Size (2023)}
\begin{itemize}
    \item Green bonds: \$1.6 trillion
    \item Sustainability-linked: \$500 billion
    \item Green loans: \$300 billion
    \item Carbon markets: \$850 billion
    \item ESG funds: \$3+ trillion AUM
\end{itemize}
\end{columns}
\end{frame}

\begin{frame}{Green Bonds: The Flagship Instrument}
\begin{columns}[T]
\column{0.48\textwidth}
\textbf{Definition}
\begin{itemize}
    \item Fixed-income securities
    \item Proceeds dedicated to green projects
    \item Same credit risk as issuer's other bonds
    \item ``Use of proceeds'' restriction
\end{itemize}

\vspace{1em}
\textbf{Eligible Project Categories}
\begin{itemize}
    \item Renewable energy
    \item Energy efficiency
    \item Clean transportation
    \item Sustainable water management
    \item Pollution prevention
    \item Green buildings
\end{itemize}

\column{0.48\textwidth}
\textbf{Key Features}
\begin{itemize}
    \item External verification (often)
    \item Regular impact reporting
    \item Separate tracking of proceeds
    \item Alignment with GBP or other standards
\end{itemize}

\vspace{1em}
\textbf{Market Growth}
\begin{itemize}
    \item 2007: First issuance (EIB, 600m EUR)
    \item 2021: Record 500 billion USD
    \item 2023: 450 billion USD
    \item Cumulative: 1.6 trillion USD outstanding
    \item Top issuers: Germany, France, US, China
\end{itemize}
\end{columns}

\small
\vspace{0.5em}
\textit{Deep dive in Week 2: Green Bond Markets and Structuring}
\end{frame}

\begin{frame}{Sustainability-Linked Instruments}
\begin{columns}[T]
\column{0.48\textwidth}
\textbf{How They Differ from Green Bonds}
\begin{itemize}
    \item \textbf{Green bonds:} Use of proceeds restriction
    \item \textbf{SL instruments:} General use, KPI-linked
    \item Financial terms tied to sustainability targets
    \item Broader issuer base (any industry)
\end{itemize}

\vspace{1em}
\textbf{Structure}
\begin{itemize}
    \item Define sustainability performance targets (SPTs)
    \item Select key performance indicators (KPIs)
    \item Set trigger events and step-up/down
    \item Typically: coupon adjustment if target missed
    \item Example: +25 bps if emissions target not met
\end{itemize}

\column{0.48\textwidth}
\textbf{Common KPIs}
\begin{itemize}
    \item GHG emissions reduction (Scope 1, 2, 3)
    \item Renewable energy share
    \item Water usage reduction
    \item Waste reduction/circularity
    \item Diversity metrics
    \item Supply chain sustainability
\end{itemize}

\vspace{1em}
\textbf{Advantages and Concerns}
\begin{itemize}
    \item Pro: Flexibility for all issuers
    \item Pro: Incentivizes corporate-wide change
    \item Con: Potential for weak targets
    \item Con: Greenwashing risk
    \item Con: KPI selection and verification challenges
\end{itemize}
\end{columns}
\end{frame}

\begin{frame}{Carbon Markets}
\begin{columns}[T]
\column{0.48\textwidth}
\textbf{Compliance Markets}
\begin{itemize}
    \item Mandatory cap-and-trade systems
    \item EU Emissions Trading System (EU ETS)
    \item California cap-and-trade
    \item Regional Greenhouse Gas Initiative (US)
    \item China national ETS
    \item UK ETS
\end{itemize}

\vspace{1em}
\textbf{EU ETS}
\begin{itemize}
    \item Covers 40\% of EU GHG emissions
    \item 10,000+ installations
    \item Carbon price: 80-100 EUR/ton (2023)
    \item Annual cap declining 2.2\% per year
    \item Market size: 700+ billion EUR
\end{itemize}

\column{0.48\textwidth}
\textbf{Voluntary Carbon Markets}
\begin{itemize}
    \item Corporate offsetting
    \item Project-based credits (VCS, Gold Standard)
    \item Nature-based solutions popular
    \item Price: 5-50 USD/ton (wide variation)
    \item Market size: 2 billion USD (2023)
\end{itemize}

\vspace{1em}
\textbf{Key Challenges}
\begin{itemize}
    \item Additionality verification
    \item Permanence concerns
    \item Leakage and double-counting
    \item Integrity of offset projects
    \item Price volatility
    \item Fraud risk in voluntary markets
\end{itemize}
\end{columns}

\small
\vspace{0.5em}
\textit{Carbon markets are a pricing mechanism but also a significant financial market with derivatives, futures, and sophisticated trading}
\end{frame}

\begin{frame}{Blended Finance Structures}
\begin{columns}[T]
\column{0.48\textwidth}
\textbf{Concept}
\begin{itemize}
    \item Strategic use of development finance
    \item Mobilize additional private capital
    \item Address market failures in sustainable finance
    \item Particularly for emerging markets
\end{itemize}

\vspace{1em}
\textbf{Financial Structures}
\begin{itemize}
    \item First-loss tranches
    \item Credit guarantees
    \item Concessional loans
    \item Technical assistance grants
    \item Political risk insurance
\end{itemize}

\column{0.48\textwidth}
\textbf{Use Cases}
\begin{itemize}
    \item Renewable energy in frontier markets
    \item Sustainable agriculture
    \item Climate adaptation infrastructure
    \item Off-grid energy access
    \item Water and sanitation projects
\end{itemize}

\vspace{1em}
\textbf{Performance Metrics}
\begin{itemize}
    \item Leverage ratio: DFI capital to private capital
    \item Typical: 1:3 to 1:10
    \item Financial returns: below market for DFI, market-rate for private
    \item Development impact: primary objective
    \item Risk-adjusted returns for private investors
\end{itemize}
\end{columns}
\end{frame}

\section{Climate Change and Finance}

\begin{frame}{The Physical Science: Why Finance Must Act}
\begin{columns}[T]
\column{0.48\textwidth}
\textbf{IPCC Key Findings}
\begin{itemize}
    \item Global warming: +1.1C above pre-industrial
    \item Unequivocal human causation
    \item 1.5C threshold likely crossed by 2030-2035
    \item 2C+ has catastrophic impacts
    \item Window for action: current decade critical
\end{itemize}

\vspace{1em}
\textbf{Investment Needs}
\begin{itemize}
    \item Paris-aligned: \$3-5 trillion/year
    \item Current investment: \$1.3 trillion/year
    \item Gap: \$2-3 trillion annually
    \item Must triple renewable energy by 2030
    \item Must halve emissions by 2030
\end{itemize}

\column{0.48\textwidth}
\textbf{Financial Implications}
\begin{itemize}
    \item Asset repricing: fossil fuels vs. renewables
    \item Stranded assets: \$1-4 trillion at risk
    \item Physical damage: \$1+ trillion/year
    \item Transition costs: front-loaded
    \item Winners: clean energy, storage, efficiency
    \item Losers: coal, oil \& gas, carbon-intensive
\end{itemize}

\vspace{1em}
\textbf{Financial System's Role}
\begin{itemize}
    \item Reallocate capital at scale
    \item Price climate risk
    \item Support innovation
    \item Manage orderly transition
\end{itemize}
\end{columns}
\end{frame}

\begin{frame}{Paris Agreement and Finance}
\begin{columns}[T]
\column{0.48\textwidth}
\textbf{Paris Agreement (2015)}
\begin{itemize}
    \item Limit warming to well below 2C, pursue 1.5C
    \item Nationally Determined Contributions (NDCs)
    \item Climate finance commitments
    \item Article 2.1c: Making finance flows consistent with low-carbon pathway
\end{itemize}

\vspace{1em}
\textbf{Article 2.1c Implications}
\begin{itemize}
    \item All finance (not just climate finance) must align
    \item Entire financial system in scope
    \item Requires climate risk integration
    \item Demands transparency and disclosure
    \item Catalyzes regulatory action
\end{itemize}

\column{0.48\textwidth}
\textbf{Climate Finance Commitments}
\begin{itemize}
    \item Developed countries: \$100 billion/year to developing
    \item Extended to 2025
    \item New goal: \$300+ billion/year post-2025
    \item Loss and damage fund established (2023)
    \item Adaptation finance: underfunded
\end{itemize}

\vspace{1em}
\textbf{Implementation Progress}
\begin{itemize}
    \item Financial sector: rapid adoption
    \item Net-zero commitments: 500+ institutions
    \item Assets: \$130+ trillion committed
    \item Challenges: credibility, transition plans
    \item Need: accountability mechanisms
\end{itemize}
\end{columns}
\end{frame}

\begin{frame}{Sustainable Development Goals and Finance}
\begin{columns}[T]
\column{0.48\textwidth}
\textbf{SDGs and Finance Linkage}
\begin{itemize}
    \item 17 goals, 169 targets (2015-2030)
    \item Investment need: \$5-7 trillion/year
    \item Current investment: insufficient
    \item SDG financing gap: \$3-4 trillion/year
    \item Green finance addresses SDGs 7, 9, 11, 12, 13, 14, 15
\end{itemize}

\vspace{1em}
\textbf{Key SDGs for Green Finance}
\begin{itemize}
    \item SDG 7: Affordable clean energy
    \item SDG 13: Climate action
    \item SDG 14: Life below water
    \item SDG 15: Life on land
\end{itemize}

\column{0.48\textwidth}
\textbf{SDG Bonds and Finance}
\begin{itemize}
    \item SDG bonds: broader than green bonds
    \item Social bonds: SDGs 1-6, 8, 10, 11, 16
    \item Sustainability bonds: environmental + social
    \item Market: 300+ billion USD/year
\end{itemize}

\vspace{1em}
\textbf{Challenges}
\begin{itemize}
    \item Too broad for targeted finance
    \item Measurement and attribution difficult
    \item Potential dilution of ``green''
    \item Need for impact taxonomy
    \item Trade-offs between goals
    \item Focus: integrate SDGs into investment frameworks
\end{itemize}
\end{columns}
\end{frame}

\section{Business Case for Green Finance}

\begin{frame}{Financial Performance of Green Investments}
\begin{columns}[T]
\column{0.48\textwidth}
\textbf{Academic Evidence}
\begin{itemize}
    \item Meta-analysis: 2000+ studies
    \item Finding: ESG does not hurt returns
    \item 58\% of studies: positive relationship
    \item 13\% of studies: negative relationship
    \item 29\% of studies: neutral relationship
    \item Time horizon matters: long-term outperformance
\end{itemize}

\vspace{1em}
\textbf{Green Bond Performance}
\begin{itemize}
    \item Return: comparable to conventional bonds
    \item Liquidity: improving but still lower
    \item Greenium: 2-5 bps (small but significant)
    \item Demand: oversubscription common
    \item Lower tail risk in some studies
\end{itemize}

\column{0.48\textwidth}
\textbf{Mechanisms for Outperformance}
\begin{itemize}
    \item Better risk management
    \item Lower cost of capital
    \item Operational efficiency
    \item Innovation capacity
    \item Regulatory preparedness
    \item Reputation and brand value
    \item Employee attraction and retention
\end{itemize}

\vspace{1em}
\textbf{Risk Considerations}
\begin{itemize}
    \item Transition risk mitigation
    \item Physical risk exposure reduction
    \item Regulatory risk management
    \item Reputational risk avoidance
    \item Future-proofing portfolios
\end{itemize}
\end{columns}

\small
\vspace{0.5em}
\textit{Key message: Green finance is financially sound, not a trade-off between returns and impact}
\end{frame}

\begin{frame}{Corporate Benefits of Green Finance}
\begin{columns}[T]
\column{0.48\textwidth}
\textbf{Cost of Capital Benefits}
\begin{itemize}
    \item Lower interest rates on green bonds
    \item Expanded investor base
    \item Reduced refinancing risk
    \item Better credit ratings (some evidence)
    \item Access to ESG-mandated capital
\end{itemize}

\vspace{1em}
\textbf{Operational Benefits}
\begin{itemize}
    \item Energy cost reduction
    \item Resource efficiency gains
    \item Waste reduction savings
    \item Process innovation
    \item Supply chain resilience
\end{itemize}

\column{0.48\textwidth}
\textbf{Strategic Benefits}
\begin{itemize}
    \item Market differentiation
    \item Customer preference alignment
    \item Regulatory compliance leadership
    \item Stakeholder relations
    \item License to operate
    \item Attract and retain talent
\end{itemize}

\vspace{1em}
\textbf{Risk Management}
\begin{itemize}
    \item Transition risk mitigation
    \item Physical risk adaptation
    \item Regulatory risk preparedness
    \item Reputational risk reduction
    \item Long-term value protection
\end{itemize}
\end{columns}
\end{frame}

\begin{frame}{Investor Motivations}
\begin{columns}[T]
\column{0.48\textwidth}
\textbf{Financial Motivations}
\begin{itemize}
    \item Risk management: climate risk exposure
    \item Return enhancement: future winners
    \item Portfolio diversification
    \item Regulatory compliance
    \item Fiduciary duty alignment
\end{itemize}

\vspace{1em}
\textbf{Non-Financial Motivations}
\begin{itemize}
    \item Values alignment
    \item Ethical considerations
    \item Mission-driven mandates
    \item Client/beneficiary preferences
    \item Reputation and brand
\end{itemize}

\column{0.48\textwidth}
\textbf{Institutional Dynamics}
\begin{itemize}
    \item Pension funds: long-term liabilities
    \item Insurance: physical risk exposure
    \item Sovereign wealth: intergenerational mandate
    \item Foundations: mission alignment
    \item Family offices: values and legacy
\end{itemize}

\vspace{1em}
\textbf{Market Evidence}
\begin{itemize}
    \item ESG fund inflows: \$500+ billion/year
    \item ESG AUM: \$30+ trillion
    \item Institutional adoption: 80\%+ considering ESG
    \item Universal owner perspective
    \item Systemic risk awareness
\end{itemize}
\end{columns}
\end{frame}

\section{Financial Fundamentals Review}

\begin{frame}{Time Value of Money Refresher}
\begin{columns}[T]
\column{0.48\textwidth}
\textbf{Core Concepts}
\begin{itemize}
    \item Present Value (PV)
    \item Future Value (FV)
    \item Discount rate (r)
    \item Number of periods (n)
\end{itemize}

\vspace{1em}
\textbf{Formulas}
\[PV = \frac{FV}{(1+r)^n}\]

\[FV = PV \times (1+r)^n\]

\[NPV = \sum_{t=0}^{n} \frac{CF_t}{(1+r)^t}\]

\column{0.48\textwidth}
\textbf{Application to Green Finance}
\begin{itemize}
    \item Long-term cash flows (renewable projects)
    \item Appropriate discount rates critical
    \item Climate risk adjusts discount rates
    \item Carbon price impacts future cash flows
    \item Regulatory changes affect timing
\end{itemize}

\vspace{1em}
\textbf{Green Finance Considerations}
\begin{itemize}
    \item Should environmental benefits be valued?
    \item Social discount rate debate
    \item Intergenerational equity
    \item Risk-free rate + climate premium?
\end{itemize}
\end{columns}
\end{frame}

\begin{frame}{Bond Pricing Fundamentals}
\begin{columns}[T]
\column{0.48\textwidth}
\textbf{Bond Price Formula}
\[P = \sum_{t=1}^{n} \frac{C}{(1+y)^t} + \frac{F}{(1+y)^n}\]

Where:
\begin{itemize}
    \item P = Price
    \item C = Coupon payment
    \item y = Yield to maturity
    \item F = Face value
    \item n = Number of periods
\end{itemize}

\vspace{1em}
\textbf{Yield Measures}
\begin{itemize}
    \item Current yield = Annual coupon / Price
    \item Yield to maturity (YTM)
    \item Yield to call (YTC)
    \item Spread over benchmark
\end{itemize}

\column{0.48\textwidth}
\textbf{Green Bond Pricing}
\begin{itemize}
    \item Same credit risk as issuer
    \item Potential greenium: slight yield discount
    \item Typically: -2 to -5 bps
    \item Demand-driven: oversubscription
    \item Liquidity considerations
\end{itemize}

\vspace{1em}
\textbf{Price Sensitivity}
\begin{itemize}
    \item Duration: price sensitivity to yield changes
    \item Longer duration: higher sensitivity
    \item Green bonds: often longer maturity
    \item Convexity: curvature of price-yield relationship
    \item Credit spread: issuer-specific risk
\end{itemize}
\end{columns}

\small
\vspace{0.5em}
\textit{Workshop: Excel exercises on bond pricing and greenium calculation}
\end{frame}

\begin{frame}{Portfolio Theory Basics}
\begin{columns}[T]
\column{0.48\textwidth}
\textbf{Key Concepts}
\begin{itemize}
    \item Expected return: $E(R_p) = \sum w_i E(R_i)$
    \item Portfolio variance: $\sigma_p^2 = \sum \sum w_i w_j \sigma_i \sigma_j \rho_{ij}$
    \item Diversification benefit
    \item Efficient frontier
    \item Capital Asset Pricing Model (CAPM)
\end{itemize}

\vspace{1em}
\textbf{Risk Measures}
\begin{itemize}
    \item Standard deviation (volatility)
    \item Beta (systematic risk)
    \item Value at Risk (VaR)
    \item Sharpe ratio: $(R_p - R_f) / \sigma_p$
\end{itemize}

\column{0.48\textwidth}
\textbf{Green Portfolio Considerations}
\begin{itemize}
    \item ESG factors as risk factors
    \item Climate risk as systematic risk
    \item Green assets: diversification benefits?
    \item Correlation with conventional assets
    \item Sector tilts: renewable energy, tech
\end{itemize}

\vspace{1em}
\textbf{Empirical Evidence}
\begin{itemize}
    \item Green portfolios: similar Sharpe ratios
    \item Lower tail risk in some studies
    \item Resilience during crises
    \item Long-term outperformance potential
    \item Need: climate-adjusted CAPM
\end{itemize}
\end{columns}

\small
\vspace{0.5em}
\textit{Workshop: Building a green portfolio in Excel, calculating risk-return metrics}
\end{frame}

\begin{frame}{Risk and Return in Green Finance}
\begin{columns}[T]
\column{0.48\textwidth}
\textbf{Return Sources}
\begin{itemize}
    \item Coupon/dividend income
    \item Capital appreciation
    \item Greenium (green bonds)
    \item Operational outperformance
    \item Risk mitigation value
    \item Regulatory tailwinds
\end{itemize}

\vspace{1em}
\textbf{Risk Factors}
\begin{itemize}
    \item Credit risk (same as conventional)
    \item Market risk (interest rate, equity)
    \item Liquidity risk (potentially higher)
    \item Greenwashing risk
    \item Technology risk (renewable energy)
    \item Policy/regulatory risk
\end{itemize}

\column{0.48\textwidth}
\textbf{Risk-Return Profile}
\begin{itemize}
    \item Green bonds: low risk, low return
    \item Listed green equities: medium risk, medium return
    \item Renewable project finance: medium-high risk, stable returns
    \item Clean tech VC: high risk, high return (potential)
    \item Emerging market green: higher risk, higher return
\end{itemize}

\vspace{1em}
\textbf{Risk Mitigation}
\begin{itemize}
    \item Diversification across technologies
    \item Geographic diversification
    \item Due diligence on green claims
    \item Third-party verification
    \item Blended finance structures
\end{itemize}
\end{columns}
\end{frame}

\section{Wrap-Up and Next Steps}

\begin{frame}{Week 1 Key Takeaways}
\begin{columns}[T]
\column{0.48\textwidth}
\textbf{Core Concepts}
\begin{enumerate}
    \item Green finance channels capital to environmental solutions
    \item Climate imperatives drive rapid market growth
    \item Diverse instruments: bonds, loans, equity, carbon markets
    \item Large and growing ecosystem of participants
\end{enumerate}

\vspace{1em}
\textbf{Financial Fundamentals}
\begin{enumerate}
    \item Green finance does not sacrifice returns
    \item Risk management is a key driver
    \item Time value of money applies with climate considerations
    \item Portfolio theory + climate risk integration
\end{enumerate}

\column{0.48\textwidth}
\textbf{Market Context}
\begin{enumerate}
    \item \$5+ trillion annual green finance market
    \item Regulatory momentum is accelerating
    \item Paris Agreement and SDGs provide framework
    \item Investment gap remains: \$2-3 trillion/year
\end{enumerate}

\vspace{1em}
\textbf{Looking Ahead}
\begin{enumerate}
    \item Week 2: Deep dive into green bonds
    \item Week 3: ESG integration and data
    \item Week 4: Climate risk assessment (TCFD)
    \item Week 5: Renewable energy project finance
    \item Week 6: Regulatory frameworks
\end{enumerate}
\end{columns}
\end{frame}

\begin{frame}{Reading Assignments}
\textbf{Required Reading (complete before Week 2):}
\begin{enumerate}
    \item Berrou et al. (2019). ``An overview of green finance'' -- foundational taxonomy
    \item Climate Bonds Initiative (2023). ``Green Bond Market Summary'' -- market data and trends
    \item UNEP (2023). ``Global Landscape of Climate Finance'' -- investment flows and gaps
\end{enumerate}

\vspace{1em}
\textbf{Recommended Reading:}
\begin{enumerate}
    \item Giglio, Kelly, and Stroebel (2021). ``Climate Finance'' -- academic review
    \item TCFD (2017). ``Final Report'' (skim for now, detailed in Week 4)
    \item ICMA (2023). ``Green Bond Principles'' (preparation for Week 2)
\end{enumerate}

\vspace{1em}
\textbf{Preparation:}
\begin{itemize}
    \item Install Python, pandas, matplotlib (if not already)
    \item Ensure Excel is functional
    \item Review provided bond pricing Excel template
\end{itemize}
\end{frame}

\begin{frame}{Workshop Exercises}
\textbf{In-Class Excel Workshops (Week 1, Session 4):}

\vspace{1em}
\textbf{Exercise 1: Bond Pricing (1 hour)}
\begin{itemize}
    \item Calculate bond prices for various yields
    \item Compute yield to maturity
    \item Analyze duration and convexity
    \item Introduce greenium concept with examples
\end{itemize}

\vspace{1em}
\textbf{Exercise 2: Portfolio Construction (1.5 hours)}
\begin{itemize}
    \item Build portfolio of 10 green and conventional assets
    \item Calculate expected return and risk
    \item Compute Sharpe ratios
    \item Explore efficient frontier
    \item Discuss diversification benefits
\end{itemize}

\vspace{1em}
\textbf{Exercise 3: NPV and IRR (1.5 hours)}
\begin{itemize}
    \item Simple renewable energy project cash flows
    \item Calculate NPV at various discount rates
    \item Compute IRR
    \item Sensitivity analysis
\end{itemize}
\end{frame}

\begin{frame}
\centering
\Huge
Questions?

\vspace{2em}
\large
Next Week: Green Bonds and Sustainable Debt Instruments

\vspace{1em}
\normalsize
See you in Week 2!
\end{frame}

\end{document}
