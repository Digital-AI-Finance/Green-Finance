\documentclass[8pt,aspectratio=169]{beamer}
\usetheme{Madrid}
\usepackage{graphicx}
\usepackage{booktabs}
\usepackage{adjustbox}
\usepackage{multicol}
\usepackage{amsmath}
\usepackage{amssymb}
\usepackage{tikz}
\usepackage{hyperref}

% Color definitions
\definecolor{mlblue}{RGB}{0,102,204}
\definecolor{mlpurple}{RGB}{51,51,178}
\definecolor{mllavender}{RGB}{173,173,224}
\definecolor{mllavender2}{RGB}{193,193,232}
\definecolor{mllavender3}{RGB}{204,204,235}
\definecolor{mllavender4}{RGB}{214,214,239}
\definecolor{mlorange}{RGB}{255, 127, 14}
\definecolor{mlgreen}{RGB}{44, 160, 44}
\definecolor{mlred}{RGB}{214, 39, 40}
\definecolor{mlgray}{RGB}{127, 127, 127}

% Apply custom colors to Madrid theme
\setbeamercolor{palette primary}{bg=mllavender3,fg=mlpurple}
\setbeamercolor{palette secondary}{bg=mllavender2,fg=mlpurple}
\setbeamercolor{palette tertiary}{bg=mllavender,fg=white}
\setbeamercolor{palette quaternary}{bg=mlpurple,fg=white}

\setbeamercolor{structure}{fg=mlpurple}
\setbeamercolor{section in toc}{fg=mlpurple}
\setbeamercolor{subsection in toc}{fg=mlblue}
\setbeamercolor{title}{fg=mlpurple}
\setbeamercolor{frametitle}{fg=mlpurple,bg=mllavender3}
\setbeamercolor{block title}{bg=mllavender2,fg=mlpurple}
\setbeamercolor{block body}{bg=mllavender4,fg=black}

% Remove navigation symbols
\setbeamertemplate{navigation symbols}{}

% Clean itemize/enumerate
\setbeamertemplate{itemize items}[circle]
\setbeamertemplate{enumerate items}[default]

% Reduce margins for more content space
\setbeamersize{text margin left=5mm,text margin right=5mm}

% Command for bottom annotation
\newcommand{\bottomnote}[1]{%
\vfill
\vspace{-2mm}
\textcolor{mllavender2}{\rule{\textwidth}{0.4pt}}
\vspace{1mm}
\footnotesize
\textbf{#1}
}

\title{Week 1 Appendix: Advanced Mathematical \& Literature Extensions}
\subtitle{Optional Material for Advanced Students}
\author{Green Finance Course}
\date{v3.2 - Appendix to Teaching-Optimized Presentation}

\begin{document}

% Title Slide
\begin{frame}[plain]
\vfill
\centering
\begin{beamercolorbox}[sep=12pt,center]{title}
{\Huge \textbf{Week 1 Appendix}}\\[1.5em]
{\Large Advanced Mathematical \& Literature Extensions}\\[1em]
{\normalsize \textcolor{mllavender}{Optional Material for Advanced Students}}
\end{beamercolorbox}
\vfill
\textit{Note: This appendix contains advanced content removed from the core presentation to reduce cognitive load. PhD-level students and those with strong economics/finance backgrounds are encouraged to study this material.}
\end{frame}

% ============================================================
% APPENDIX A: FORMAL SEGMENTATION MODEL
% ============================================================

\begin{frame}[t]{Appendix A: Market Segmentation - Formal Equilibrium Model}
\begin{center}
{\large \textbf{Advanced Mathematical Extension to Slide 8}}
\end{center}

\vspace{0.5em}

\begin{columns}[T]
\column{0.48\textwidth}
\textbf{Model Setup}
\begin{itemize}
\item Two investor types: ESG-preferring ($\lambda$), Conventional (1-$\lambda$)
\item Utility functions:
  \begin{align*}
  U_E(r) &= r + \alpha \cdot g \quad (\alpha > 0)\\
  U_C(r) &= r
  \end{align*}
\item $g \in \{0,1\}$ = green label, $\alpha$ = ESG preference intensity
\item Supply: $S_G$ green bonds, $S_C$ conventional bonds
\end{itemize}

\column{0.48\textwidth}
\textbf{Equilibrium Conditions}
\begin{itemize}
\item Market clearing: $\lambda \cdot D_E(r_G) = S_G$, $(1-\lambda) \cdot D_C(r_C) = S_C$
\item Greenium emerges: $r_C - r_G = \frac{\alpha \cdot \lambda}{D'(r)}$ if $S_G < \lambda \cdot D(r_C)$
\item \textbf{Key prediction}: Greenium $\propto$ ESG investor share ($\lambda$)
\item \textbf{Testable}: Greenium larger in EU (high $\lambda$) than US
\item \textbf{Dynamic}: As $S_G \uparrow$, greenium $\downarrow$ (Slide 34 confirms)
\end{itemize}
\end{columns}

\vspace{0.5em}

\textbf{Derivation Notes:}
\begin{itemize}
\item Assumes constant elasticity of demand: $D(r) = A \cdot r^{-\eta}$ where $\eta > 0$
\item ESG investor indifference condition: $r_G + \alpha = r_C$ (willing to accept lower return)
\item Greenium magnitude depends on: (1) preference intensity $\alpha$, (2) ESG investor fraction $\lambda$, (3) relative supply $S_G/S_C$
\item Empirical calibration: $\alpha \approx 0.0003$ (3 bps), $\lambda_{EU} \approx 0.35$, $\lambda_{US} \approx 0.15$ generates observed regional differences
\end{itemize}

\bottomnote{[Appendix A] Formal model generates testable hypotheses - validated by regional and temporal variation data (Slides 18, 34)}
\end{frame}

% ============================================================
% APPENDIX B: ACADEMIC LITERATURE REVIEW
% ============================================================

\begin{frame}[t]{Appendix B: Academic Literature - Empirical Evidence on Green Bonds}
\begin{center}
{\large \textbf{Comprehensive Literature Summary for Further Reading}}
\end{center}

\vspace{0.5em}

\begin{columns}[T]
\column{0.48\textwidth}
\textbf{Greenium Existence (Pricing)}
\begin{itemize}
\item \textbf{Zerbib (2019)}: -2 bps YTM for green bonds
\item \textbf{Baker et al. (2018)}: 6 bps greenium in US municipals, larger with certification
\item \textbf{Karpf \& Mandel (2018)}: 5-9 bps, time-varying
\item \textbf{Ando (2024)}: 11 bps emerging market sovereigns, 2 bps advanced
\item \textbf{Consensus}: Greenium exists but varies by market segment and time
\end{itemize}

\column{0.48\textwidth}
\textbf{Corporate Impact (Shareholder Value)}
\begin{itemize}
\item \textbf{Flammer (2021)}: +0.5\% stock return on green bond announcement; increased green patents
\item \textbf{Tang \& Zhang (2020)}: Positive wealth effects, esp. in polluting industries
\item \textbf{Additionality}: Mixed evidence - some funding truly new projects, some relabeling
\end{itemize}

\vspace{0.3em}

\textbf{Financial Institution Role}
\begin{itemize}
\item \textbf{Fatica et al. (2021)}: Banks pay higher greenium (-9 bps) due to reputational concerns
\item \textbf{Implication}: Verification more critical for repeat issuers
\end{itemize}
\end{columns}

\vspace{0.5em}

\textbf{Research frontier}: Impact measurement methodology, long-run greenium dynamics, optimal policy mix.

\vspace{0.3em}

\textbf{Recommended Reading Order:}
\begin{enumerate}
\item Start with Zerbib (2019) for greenium measurement methodology
\item Read Flammer (2021) for corporate impact evidence
\item Explore Ando (2024) for recent sovereign bond findings
\item Advanced: Baker et al. (2018) for theoretical segmentation model
\end{enumerate}

\bottomnote{[Appendix B] Academic literature 2018-2024 provides robust empirical support for segmentation and signaling theories - See References for full citations}
\end{frame}

% References for Appendix
\begin{frame}[t]{Appendix References}
\tiny

\textbf{Key Papers Discussed in Appendix B:}

\begin{itemize}
\item Zerbib, O.D. (2019). The Effect of Pro-Environmental Preferences on Bond Prices: Evidence from Green Bonds. \textit{Journal of Banking \& Finance}, 98, 39-60. doi:10.1016/j.jbankfin.2018.10.012

\item Baker, M., Bergstresser, D., Serafeim, G., \& Wurgler, J. (2018). Financing the Response to Climate Change: The Pricing and Ownership of U.S. Green Bonds. \textit{NBER Working Paper}, 25194. doi:10.3386/w25194

\item Karpf, A., \& Mandel, A. (2018). The Changing Value of the 'Green' Label on the US Municipal Bond Market. \textit{Nature Climate Change}, 8, 161-165. doi:10.1038/s41558-017-0062-0

\item Ando, S., \& Greenwood-Nimmo, M. (2024). How Large is the Sovereign Greenium?. \textit{Oxford Bulletin of Economics and Statistics}, 86(3), 594-621. doi:10.1111/obes.12619

\item Flammer, C. (2021). Corporate Green Bonds. \textit{Journal of Financial Economics}, 142(2), 499-516. doi:10.1016/j.jfineco.2021.01.010

\item Tang, D.Y., \& Zhang, Y. (2020). Do Shareholders Benefit from Green Bonds?. \textit{Journal of Corporate Finance}, 61, 101427. doi:10.1016/j.jcorpfin.2018.12.001

\item Fatica, S., Panzica, R., \& Rancan, M. (2021). The Pricing of Green Bonds: Are Financial Institutions Special?. \textit{Journal of Financial Stability}, 54, 100873. doi:10.1016/j.jfs.2021.100873
\end{itemize}

\vspace{0.5em}

\textbf{Note:} Complete reference list available in main presentation.

\end{frame}

\end{document}
