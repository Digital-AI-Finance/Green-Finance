\documentclass[8pt,aspectratio=169]{beamer}
\usetheme{Madrid}
\usepackage{graphicx}
\usepackage{booktabs}
\usepackage{adjustbox}
\usepackage{multicol}
\usepackage{amsmath}

% Color definitions
\definecolor{mlblue}{RGB}{0,102,204}
\definecolor{mlpurple}{RGB}{51,51,178}
\definecolor{mllavender}{RGB}{173,173,224}
\definecolor{mllavender2}{RGB}{193,193,232}
\definecolor{mllavender3}{RGB}{204,204,235}
\definecolor{mllavender4}{RGB}{214,214,239}
\definecolor{mlorange}{RGB}{255, 127, 14}
\definecolor{mlgreen}{RGB}{44, 160, 44}
\definecolor{mlred}{RGB}{214, 39, 40}
\definecolor{mlgray}{RGB}{127, 127, 127}
\definecolor{lightgray}{RGB}{240, 240, 240}
\definecolor{midgray}{RGB}{180, 180, 180}

% Apply custom colors to Madrid theme
\setbeamercolor{palette primary}{bg=mllavender3,fg=mlpurple}
\setbeamercolor{palette secondary}{bg=mllavender2,fg=mlpurple}
\setbeamercolor{palette tertiary}{bg=mllavender,fg=white}
\setbeamercolor{palette quaternary}{bg=mlpurple,fg=white}
\setbeamercolor{structure}{fg=mlpurple}
\setbeamercolor{section in toc}{fg=mlpurple}
\setbeamercolor{subsection in toc}{fg=mlblue}
\setbeamercolor{title}{fg=mlpurple}
\setbeamercolor{frametitle}{fg=mlpurple,bg=mllavender3}
\setbeamercolor{block title}{bg=mllavender2,fg=mlpurple}
\setbeamercolor{block body}{bg=mllavender4,fg=black}

% Remove navigation symbols
\setbeamertemplate{navigation symbols}{}
\setbeamertemplate{itemize items}[circle]
\setbeamertemplate{enumerate items}[default]
\setbeamersize{text margin left=5mm,text margin right=5mm}

% Command for bottom annotation
\newcommand{\bottomnote}[1]{%
\vfill
\vspace{-2mm}
\textcolor{mllavender2}{\rule{\textwidth}{0.4pt}}
\vspace{1mm}
\footnotesize
\textbf{#1}
}

\title{Week 1: Green Finance Foundations}
\subtitle{Professional Certificate in Green Finance}
\author{}
\date{}

\begin{document}

% ==================== TITLE SLIDE ====================
\begin{frame}[plain]
\titlepage
\end{frame}

% ==================== TABLE OF CONTENTS ====================
\begin{frame}[t]{Week 1 Overview}
\tableofcontents
\bottomnote{14 contact hours | Sessions: Intro, Ecosystem, Instruments, Financial Fundamentals}
\end{frame}

% ==================== SECTION 1: INTRODUCTION ====================
\section{Introduction to Green Finance}

\begin{frame}[t]
\vfill
\centering
\begin{beamercolorbox}[sep=8pt,center]{title}
\usebeamerfont{title}\Large Introduction to Green Finance\par
\end{beamercolorbox}
\vfill
\end{frame}

% ==================== SLIDE: WHAT IS GREEN FINANCE ====================
\begin{frame}[t]{What is Green Finance?}
\begin{columns}[T]
\column{0.48\textwidth}
\textbf{Definition}
\begin{itemize}
\item Financial investments supporting environmental sustainability
\item Capital directed to climate action and ecological objectives
\item Integration of environmental criteria into financial decisions
\item Risk-adjusted returns with measurable impact
\end{itemize}

\vspace{0.5em}
\textbf{Key Characteristics}
\begin{itemize}
\item Environmental additionality
\item Transparency and reporting
\item Third-party verification
\end{itemize}

\column{0.48\textwidth}
\textbf{Scope of Green Finance}
\begin{itemize}
\item Climate change mitigation
\item Climate change adaptation
\item Pollution prevention and control
\item Biodiversity conservation
\item Sustainable resource use
\item Circular economy
\end{itemize}

\vspace{0.5em}
\textbf{Market Scale (2024)}
\begin{itemize}
\item Total: 5+ trillion USD annually
\item Green bonds: 1.6 trillion USD outstanding
\item Growth: 30-40\% per year
\end{itemize}
\end{columns}
\bottomnote{Green finance channels capital to activities with positive environmental outcomes while maintaining financial viability}
\end{frame}

% ==================== CHART: MARKET GROWTH ====================
\begin{frame}[t]{Green Finance Market Growth Trajectory}
\begin{center}
\includegraphics[width=0.85\textwidth]{charts/week1/market_growth.pdf}
\end{center}
\bottomnote{Source: Climate Bonds Initiative, UNEP. Market compound annual growth rate: 31\% (2015-2024)}
\end{frame}

% ==================== SLIDE: WHY IT MATTERS ====================
\begin{frame}[t]{Why Green Finance Matters}
\begin{columns}[T]
\column{0.48\textwidth}
\textbf{Climate Imperative}
\begin{itemize}
\item Paris Agreement: limit warming to 1.5-2C
\item Required investment: 3-5 trillion USD/year
\item Current gap: 2-3 trillion USD/year
\item Financial system must mobilize capital
\end{itemize}

\vspace{0.5em}
\textbf{Financial Risk}
\begin{itemize}
\item Physical risks: extreme weather, rising seas
\item Transition risks: policy, technology shifts
\item Stranded assets: 1-4 trillion USD at risk
\item Systemic financial stability concerns
\end{itemize}

\column{0.48\textwidth}
\textbf{Business Opportunity}
\begin{itemize}
\item Clean energy market: 10+ trillion USD
\item First-mover advantages
\item Innovation in sustainable tech
\item Growing investor demand
\end{itemize}

\vspace{0.5em}
\textbf{Regulatory Drivers}
\begin{itemize}
\item EU Taxonomy and SFDR
\item SEC climate disclosure rules
\item Central bank climate stress tests
\item Mandatory TCFD reporting globally
\end{itemize}
\end{columns}
\bottomnote{Green finance addresses climate urgency while creating significant economic opportunities and managing financial risks}
\end{frame}

% ==================== SLIDE: HISTORICAL DEVELOPMENT ====================
\begin{frame}[t]{Evolution of Green Finance}
\begin{columns}[T]
\column{0.48\textwidth}
\textbf{Early History (Pre-2007)}
\begin{itemize}
\item 1990s: Socially responsible investing (SRI)
\item 2000: UN Global Compact launched
\item 2006: UN Principles for Responsible Investment
\end{itemize}

\vspace{0.5em}
\textbf{Emergence (2007-2015)}
\begin{itemize}
\item 2007: First green bond (EIB, 600m EUR)
\item 2014: Green Bond Principles
\item 2015: Paris Agreement catalyst
\end{itemize}

\column{0.48\textwidth}
\textbf{Mainstreaming (2015-2020)}
\begin{itemize}
\item 2017: TCFD recommendations
\item Explosive green bond market growth
\item Central banks engage climate risk
\end{itemize}

\vspace{0.5em}
\textbf{Maturation (2020-Present)}
\begin{itemize}
\item 2021: EU Taxonomy implemented
\item 2022: SFDR disclosure requirements
\item 2024: Regulatory frameworks solidify
\item Focus shifts to impact and credibility
\end{itemize}
\end{columns}
\bottomnote{Green finance evolved from niche ethical investing to mainstream financial practice driven by climate science and regulation}
\end{frame}

% ==================== SECTION 2: ECOSYSTEM ====================
\section{Green Finance Ecosystem}

\begin{frame}[t]
\vfill
\centering
\begin{beamercolorbox}[sep=8pt,center]{title}
\usebeamerfont{title}\Large Green Finance Ecosystem\par
\end{beamercolorbox}
\vfill
\end{frame}

% ==================== CHART: ECOSYSTEM MAP ====================
\begin{frame}[t]{Green Finance Ecosystem Map}
\begin{center}
\includegraphics[width=0.75\textwidth]{charts/week1/ecosystem_map.pdf}
\end{center}
\bottomnote{Capital flows from providers through intermediaries to recipients, guided by standard-setters and regulators}
\end{frame}

% ==================== SLIDE: MARKET PARTICIPANTS ====================
\begin{frame}[t]{Key Market Participants}
\begin{columns}[T]
\column{0.48\textwidth}
\textbf{Capital Providers}
\begin{itemize}
\item Institutional investors (pensions, insurance)
\item Asset managers and funds
\item Commercial banks
\item Development finance institutions
\item Retail investors
\end{itemize}

\vspace{0.5em}
\textbf{Intermediaries}
\begin{itemize}
\item Investment banks (underwriting)
\item Rating agencies
\item Verifiers and certifiers
\item Stock exchanges
\end{itemize}

\column{0.48\textwidth}
\textbf{Capital Recipients}
\begin{itemize}
\item Sovereign governments
\item Corporations (green bonds, loans)
\item Project developers (renewable energy)
\item Municipalities
\item Financial institutions
\end{itemize}

\vspace{0.5em}
\textbf{Standard-Setters}
\begin{itemize}
\item ICMA (Green Bond Principles)
\item Climate Bonds Initiative
\item ISSB (sustainability standards)
\item EU Commission (Taxonomy)
\end{itemize}
\end{columns}
\bottomnote{Diverse participants create a complex ecosystem requiring coordination through standards and regulations}
\end{frame}

% ==================== SECTION 3: INSTRUMENTS ====================
\section{Green Financial Instruments}

\begin{frame}[t]
\vfill
\centering
\begin{beamercolorbox}[sep=8pt,center]{title}
\usebeamerfont{title}\Large Green Financial Instruments\par
\end{beamercolorbox}
\vfill
\end{frame}

% ==================== CHART: INSTRUMENT BREAKDOWN ====================
\begin{frame}[t]{Green Finance Instruments Overview}
\begin{center}
\includegraphics[width=0.95\textwidth]{charts/week1/instrument_breakdown.pdf}
\end{center}
\bottomnote{Green bonds dominate the market at 1.6 trillion USD, followed by carbon markets and sustainability-linked instruments}
\end{frame}

% ==================== SLIDE: GREEN BONDS ====================
\begin{frame}[t]{Green Bonds: The Flagship Instrument}
\begin{columns}[T]
\column{0.48\textwidth}
\textbf{Definition and Structure}
\begin{itemize}
\item Fixed-income securities
\item Proceeds dedicated to green projects
\item Same credit risk as issuer
\item Use-of-proceeds restriction
\end{itemize}

\vspace{0.5em}
\textbf{Eligible Categories}
\begin{itemize}
\item Renewable energy
\item Energy efficiency
\item Clean transportation
\item Green buildings
\item Sustainable water management
\end{itemize}

\column{0.48\textwidth}
\textbf{Key Features}
\begin{itemize}
\item External verification (common)
\item Regular impact reporting
\item Separate tracking of proceeds
\item Alignment with GBP or standards
\end{itemize}

\vspace{0.5em}
\textbf{Market Size}
\begin{itemize}
\item First issuance: 2007 (EIB, 600m EUR)
\item 2023: 450 billion USD issued
\item Outstanding: 1.6 trillion USD
\item Top issuers: Germany, France, US, China
\end{itemize}
\end{columns}
\bottomnote{Green bonds provide transparent, verifiable financing for environmental projects with market-rate returns}
\end{frame}

% ==================== SLIDE: SUSTAINABILITY-LINKED ====================
\begin{frame}[t]{Sustainability-Linked Instruments}
\begin{columns}[T]
\column{0.48\textwidth}
\textbf{Key Difference from Green Bonds}
\begin{itemize}
\item No use-of-proceeds restriction
\item General corporate use allowed
\item Financial terms tied to KPIs
\item Broader issuer base (any industry)
\end{itemize}

\vspace{0.5em}
\textbf{Structure}
\begin{itemize}
\item Define Sustainability Performance Targets
\item Select Key Performance Indicators
\item Coupon step-up if targets missed
\item Example: +25 bps if emissions target unmet
\end{itemize}

\column{0.48\textwidth}
\textbf{Common KPIs}
\begin{itemize}
\item GHG emissions reduction (Scope 1, 2, 3)
\item Renewable energy share
\item Water usage reduction
\item Waste reduction / circularity
\end{itemize}

\vspace{0.5em}
\textbf{Benefits and Concerns}
\begin{itemize}
\item Pro: Incentivizes corporate-wide change
\item Pro: Flexible for all sectors
\item Con: Potential for weak targets
\item Con: Greenwashing risk
\end{itemize}
\end{columns}
\bottomnote{Sustainability-linked bonds shift focus from project-level to entity-level performance commitments}
\end{frame}

% ==================== SLIDE: CARBON MARKETS ====================
\begin{frame}[t]{Carbon Markets Overview}
\begin{columns}[T]
\column{0.48\textwidth}
\textbf{Compliance Markets}
\begin{itemize}
\item Mandatory cap-and-trade systems
\item EU ETS: 700+ billion EUR market
\item California cap-and-trade
\item China national ETS
\item Price: 80-100 EUR/ton (EU ETS 2024)
\end{itemize}

\vspace{0.5em}
\textbf{EU ETS Details}
\begin{itemize}
\item Covers 40\% of EU GHG emissions
\item 10,000+ installations
\item Cap declining 2.2\% per year
\end{itemize}

\column{0.48\textwidth}
\textbf{Voluntary Carbon Markets}
\begin{itemize}
\item Corporate offsetting
\item Project-based credits (VCS, Gold Standard)
\item Nature-based solutions popular
\item Price: 5-50 USD/ton (high variance)
\item Market size: 2 billion USD (2024)
\end{itemize}

\vspace{0.5em}
\textbf{Key Challenges}
\begin{itemize}
\item Additionality verification
\item Permanence concerns
\item Double-counting risks
\item Integrity of offset projects
\end{itemize}
\end{columns}
\bottomnote{Carbon markets provide price signals for emissions reductions but face credibility and quality challenges}
\end{frame}

% ==================== SECTION 4: FINANCIAL FUNDAMENTALS ====================
\section{Financial Fundamentals Review}

\begin{frame}[t]
\vfill
\centering
\begin{beamercolorbox}[sep=8pt,center]{title}
\usebeamerfont{title}\Large Financial Fundamentals Review\par
\end{beamercolorbox}
\vfill
\end{frame}

% ==================== SLIDE: TIME VALUE OF MONEY ====================
\begin{frame}[t]{Time Value of Money in Green Finance}
\begin{columns}[T]
\column{0.48\textwidth}
\textbf{Core Concepts}
\begin{itemize}
\item Present Value (PV)
\item Future Value (FV)
\item Discount rate (r)
\item Number of periods (n)
\end{itemize}

\vspace{0.5em}
\textbf{Formulas}
$$PV = \frac{FV}{(1+r)^n}$$

$$NPV = \sum_{t=0}^{n} \frac{CF_t}{(1+r)^t}$$

\column{0.48\textwidth}
\textbf{Application to Green Finance}
\begin{itemize}
\item Long-term cash flows (renewable projects)
\item Appropriate discount rates critical
\item Climate risk adjusts discount rates
\item Carbon pricing impacts future cash flows
\end{itemize}

\vspace{0.5em}
\textbf{Green Finance Considerations}
\begin{itemize}
\item Should environmental benefits be valued?
\item Social discount rate debate
\item Intergenerational equity
\item Risk-free rate + climate premium?
\end{itemize}
\end{columns}
\bottomnote{Time value of money principles apply to green finance but require adjustment for long horizons and climate uncertainty}
\end{frame}

% ==================== SLIDE: BOND PRICING ====================
\begin{frame}[t]{Bond Pricing and Greenium}
\begin{columns}[T]
\column{0.48\textwidth}
\textbf{Bond Price Formula}
$$P = \sum_{t=1}^{n} \frac{C}{(1+y)^t} + \frac{F}{(1+y)^n}$$

Where:
\begin{itemize}
\item P = Price
\item C = Coupon payment
\item y = Yield to maturity
\item F = Face value
\end{itemize}

\vspace{0.5em}
\textbf{Yield Measures}
\begin{itemize}
\item Yield to maturity (YTM)
\item Spread over benchmark
\end{itemize}

\column{0.48\textwidth}
\textbf{Green Bond Pricing}
\begin{itemize}
\item Same credit risk as issuer
\item Potential greenium: -2 to -5 bps
\item Demand-driven oversubscription
\item Liquidity considerations
\end{itemize}

\vspace{0.5em}
\textbf{Price Sensitivity}
\begin{itemize}
\item Duration: sensitivity to yield changes
\item Green bonds: often longer maturity
\item Convexity: price-yield curvature
\item Credit spread: issuer risk
\end{itemize}
\end{columns}
\bottomnote{Green bonds price similarly to conventional bonds but exhibit small yield discount (greenium) due to excess demand}
\end{frame}

% ==================== SLIDE: PORTFOLIO THEORY ====================
\begin{frame}[t]{Portfolio Theory and Green Investing}
\begin{columns}[T]
\column{0.48\textwidth}
\textbf{Key Concepts}
\begin{itemize}
\item Expected return: $E(R_p) = \sum w_i E(R_i)$
\item Portfolio risk (variance)
\item Diversification benefit
\item Efficient frontier
\item CAPM framework
\end{itemize}

\vspace{0.5em}
\textbf{Risk Measures}
\begin{itemize}
\item Standard deviation (volatility)
\item Beta (systematic risk)
\item Sharpe ratio: $(R_p - R_f) / \sigma_p$
\end{itemize}

\column{0.48\textwidth}
\textbf{Green Portfolio Considerations}
\begin{itemize}
\item ESG factors as risk factors
\item Climate risk as systematic risk
\item Green assets: diversification benefits
\item Sector tilts: renewable energy, tech
\end{itemize}

\vspace{0.5em}
\textbf{Empirical Evidence}
\begin{itemize}
\item Similar Sharpe ratios to conventional
\item Lower tail risk in some studies
\item Resilience during crises
\item Long-term outperformance potential
\end{itemize}
\end{columns}
\bottomnote{Modern portfolio theory applies to green finance with climate risk as additional systematic factor requiring integration}
\end{frame}

% ==================== SLIDE: WEEK 1 SUMMARY ====================
\begin{frame}[t]{Week 1 Key Takeaways}
\begin{columns}[T]
\column{0.48\textwidth}
\textbf{Core Concepts}
\begin{itemize}
\item Green finance mobilizes capital for environmental outcomes
\item Market size: 5+ trillion USD annually
\item Growth driven by climate urgency and regulation
\item Diverse ecosystem of participants
\end{itemize}

\vspace{0.5em}
\textbf{Key Instruments}
\begin{itemize}
\item Green bonds: 1.6 trillion USD market
\item Sustainability-linked bonds
\item Carbon markets
\end{itemize}

\column{0.48\textwidth}
\textbf{Financial Fundamentals}
\begin{itemize}
\item Time value of money applies with adjustments
\item Green bonds price with small greenium
\item Portfolio theory + climate risk integration
\item No return sacrifice for green investing
\end{itemize}

\vspace{0.5em}
\textbf{Looking Ahead}
\begin{itemize}
\item Week 2: Deep dive into green bonds
\item Week 3: ESG integration and analytics
\item Week 4: Climate risk assessment (TCFD)
\end{itemize}
\end{columns}
\bottomnote{Green finance is financially sound, rapidly growing, and essential for climate transition with significant business opportunities}
\end{frame}

% ==================== CLOSING SLIDE ====================
\begin{frame}[plain]
\vspace{3cm}
\begin{center}
{\Large Week 1 Complete}\\[2cm]
{\normalsize Next: Week 2 - Green Bonds and Sustainable Debt Instruments}\\[1cm]
{\small Reading assignments distributed | Prepare Excel for workshops}
\end{center}
\end{frame}

\end{document}
